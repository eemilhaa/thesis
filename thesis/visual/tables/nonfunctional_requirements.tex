\begin{table}[H]
	\caption{
		The nonfunctional requirements of the map application specify
		its desired qualities (\ref{tab:quality attributes}) and
		the constraints it must adhere to (\ref{tab:constraints}).
	}
	\label{tab:nonfunctional requirements}
	\begin{subtable}[h]{\textwidth}
		\caption{}
		\label{tab:quality attributes}
		\centering
		\begin{tabular}{ | L{0.2\textwidth} | L{0.7\textwidth} | }
			\hline
			\textbf{Category}
			& \textbf{Requirements}
			\\
			\hline
			\hline
			Performance
			& \tabitem Data serving speed allows for real-time interaction. \\
			& \tabitem Map rendering speed allows for real-time interaction. \\
			\hline
			Maintainability
			& \tabitem All components are as independent as possible. \\
			& \tabitem The codebase is versioned and documented. \\
			& \tabitem Deploying the application is reproducible. \\
			\hline
			Usability
			& \tabitem Visual feedback from user interaction is instantaneous. \\
			\hline
			Scalability
			& \tabitem The application is scalable to meet different usage loads. \\
			& \tabitem Different application components can be scaled independently. \\
			\hline
		\end{tabular}
	\end{subtable}
	\newline
	\newline  % https://tex.stackexchange.com/questions/38893/cant-generate-vertical-space-between-tables
	\newline
	\begin{subtable}[h]{\textwidth}
		\caption{}
		\label{tab:constraints}
		\centering
		\begin{tabular}{ | L{0.2\textwidth} | L{0.7\textwidth} | }
			\hline
			\textbf{Type of constraint}
			& \textbf{Description}
			\\
			\hline
			\hline
			Client-side platform
			& The frontend of the map application runs in a web browser.
			\\
			\hline
			Deployment environment
			& The front and backend are deployed in containers
			utilizing the OpenShift container platform.
			\\
			\hline
		\end{tabular}
	\end{subtable}
\end{table}
