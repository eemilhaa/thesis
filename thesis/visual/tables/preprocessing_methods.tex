\begin{table}[H]
	\caption{The preprocessing methods}
	\label{tab:preprocessing methods}
	\centering
	\begin{tabular}{ | L{0.21\textwidth} | L{0.35\textwidth} | L{0.35\textwidth} | }
		\hline
		\textbf{Method}
		& \textbf{Increase in performance}
		& \textbf{Loss of information}
		\\
		\hline
		\hline
		Aggregation into isochrone polygons (15-minute interval)
		& Large: The isochronal approach is what makes the real-time interaction possible.
		& Large: 15-minute isochrone polygons allow for quick overview but make detailed assessment impossible.
		\\
		\hline
		Limiting the maximum travel time (60 minutes)
		& Noticeable: The largest isochrone polygons are more costly to render.
		Avoiding them improves performance, and on limited hardware / download speed
		is essential to keep the map performant.
		& Depends on travel mode:
		For walking a limit of 60 minutes means that, on average,
		the isochrones cover 3\% of the total area of the dataset.
		For cycling 30\%, for public transit 27\%, for car 98\%.
		For a given grid cell the averaged coverage of all modes is 40\%.
		\\
		\hline
		Limiting coordinate precision
		& Potentially noticeable with limited download speeds:
		Reducing the precision of geometries reduced file sizes by 50\% on average,
		thus enabling faster data transfer.
		No visible effect on rendering performance on tested hardware.
		& None: The decreased precision is in no way visible.
		\\
		\hline
		File optimization (simplifying GeoJSON structure, compression using gzip)
		& Potentially noticeable with limited download speeds:
		This reduces file sizes by 60\% on average,  % TODO
		thus enabling faster data transfer.
		No visible effect on rendering performance on tested hardware.
		& None: These approaches do not affect the content of the data.
		\\
		\hline
	\end{tabular}
\end{table}
