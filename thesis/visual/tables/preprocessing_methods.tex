\begin{table}[H]
	\caption{
		The assessment of the preprocessing methods.
		The methods applying the highest levels of geometrical simplification
		resulted in the largest increases in map responsiveness,
		but also resulted in significant information loss when mapping the data.
	}
	\label{tab:preprocessing methods}
	\centering
	\begin{tabular}{ | L{0.25\textwidth} | L{0.33\textwidth} | L{0.33\textwidth} | }
		\hline
		\textbf{Method, type of optimization}
		& \textbf{Increase in responsiveness of the map}
		& \textbf{Loss of information on the map}
		\\
		\hline
		\hline
		Aggregation into isochrone polygons (15-minute interval),
		\textit{Reducing geometrical complexity and file sizes}.
		& Large: The isochronal approach is what makes the real-time interaction possible.
		& Large: 15-minute isochrone polygons allow for quick overview
		but make detailed assessment impossible.
		\\
		\hline
		Limiting the maximum travel time (60 minutes),
		\textit{Reducing geometrical complexity and file sizes}.
		& Noticeable on all travel modes, as the largest isochrone polygons are more costly to render.
		The more data was discarded, the larger the increase:
		most increase observed when mapping the slowest travel modes.
		& Depends on travel mode:
		For walking a limit of 60 minutes means that, on average,
		the isochrones cover 3\% of the total area of the dataset.
		For cycling 30\%, for public transit 27\%, for car 98\%.
		For a given grid cell the averaged coverage of all modes is 40\%.
		\\
		\hline
		Limiting coordinate precision,
		\textit{Reducing file sizes}.
		& Potentially noticeable with limited download speeds:
		Reducing the precision of geometries reduced file sizes by 50\% on average,
		thus enabling faster data transfer.
		No visible effect on responsiveness was observed in this study.
		& None: The decreased precision is in no way visible.
		\\
		\hline
		Minimizing files by simplifying GeoJSON structure and compressing using gzip,
		\textit{Reducing file sizes}.
		& Potentially noticeable with limited download speeds:
		This reduces file sizes by 60\% on average,  % TODO
		thus enabling faster data transfer.
		No visible effect on responsiveness was observed in this study.
		& None: These approaches do not affect the content of the data.
		\\
		\hline
	\end{tabular}
\end{table}
