\begin{table}[H]
	\caption{The preprocessing methods used}
	\label{tab:preprocessing methods}
	\centering
	\begin{tabular}{ | L{0.2\textwidth} | L{0.35\textwidth} | L{0.35\textwidth} | }
		\hline
		Method
		& Impact on performance
		& Impact on visual detail
		\\
		\hline
		\hline
		Aggregation into isochrones (15 min interval)
		& Large; The isochronal approach is what makes the real-time interaction possible.
		& Large; Isochrones allow for quick overview but make detailed assessment difficult.
		\\
		\hline
		Limiting the maximum travel time (60 min)
		& Noticeable; The largest isochrone polygons are more costly to render.
		Avoiding them improves performance especially on limited hardware.
		& Noticeable depending on travel mode;
		The maximum travel time of 60 minutes drops a lot of information for slower travel modes (walk, bike),  % TODO percentages
		while faster modes (public transit, car) do not lose much information.
		\\
		\hline
		Geometrical simplification
		& Noticeable with limited download speeds;
		Reducing the precision of geometry reduces file sizes approximately by (percentage),  % TODO
		and thus makes data transfer faster.
		No visible effect on rendering performance on tested hardware.
		& None; The decreased precision is in no way visible.
		\\
		\hline
		File optimization (minimal file structure, compression)
		& Noticeable with limited download speeds;
		This reduces file sizes by (percentage),  % TODO
		and thus makes data transfer faster.
		No visible effect on rendering performance on tested hardware.
		& None; These approaches do not affect the content of the data
		\\
		\hline
	\end{tabular}
\end{table}
