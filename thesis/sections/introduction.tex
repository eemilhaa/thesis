\section{Introduction}
% Start with cartography instead? (kra2021)

% Cartography
% importance: recent developments, data, revealing patterns. See also \textcite{kra2021}.

% Accessibility shortly
% Accessibility from a cartographical angle
% challenges in presenting

\subsection{Motivation}

Maps are ubiquitous and essential to human civilization.
In addition to depicting our geographical surroundings,
they help us reveal and understand new patterns and phenomena,
and to gain insights from information
that would otherwise be hard or impossible to interpret \parencite{mac2004}.
Currently, we live in a world that is in a state of rapid and
alarming change \parencite{un2023},
but where information is available in an unprecedented abundance.
Acknowledged as one of the most pressing issues of the ongoing decade,
efficiently harnessing this data is a requirement
for the well-being of the entire planet \parencite{un2020}.
As the quantity of data grows
and understanding it becomes more and more demanding,
the relevance of maps and cartography is only increasing.
Maps, mapping technologies, and cartography are essential,
as they can provide unique ways to synthesize and make sense of complex data,
of which there is no shortage \parencite{kra2021}.
% To summarize, any means that allow us to make sense of an overflow of information are vital,
% and maps and cartography certainly are such means.
% and it is no coincidence that \textcite{kra2021}

% TODO? Cartography in change
% new methods and new topics unlocked
% Much like the world it depicts, cartography itself is changing too.


% This thesis applies all this to accessibility
In this thesis I relate cartography to accessibility.
When referring to accessibility,
I more specifically mean spatial accessibility, that is,
the ease of reaching valued destinations in physical space \parencite{lev2020}.
Understanding accessibility is important,
as the phenomenon is deeply interlinked in our society.
Accessibility affects the choices and experiences of individuals \parencite{kwa1998, kwa2003}
as well as the planning and governance processes
of much larger entities such as cities \parencite{cur2010, low2015}.
It is also directly linked to significant and ongoing issues such as
urban well-being \parencite{zha2011},
equity and justice \parencite{per2017, che2020}
and sustainable transport and mobility \parencite{son2017, mah2019}.
In general, research in accessibility helps us understand
how societal systems and phenomena function
in different scales and timeframes, and from different points of view.

% difficulty in defining and measuring access
However, analysing or even comprehending accessibility
in a holistic manner is not simple.
While a common starting point for defining the term is, quite concisely,
\enquote{potential of opportunities for interaction} \parencite{han1959},  % define interaction in more detail?
more precise definitions and measurements can be constructed in countless ways
depending on the factors deemed impactful to that potential
\parencite{pap2016}.
For example, even with a simple consideration of travel distance as a measure of access,
including all the different things this access can be measured in relation to
would make the amount of variations, i.e. different \enquote{accessibilities},
to measure grow exponentially \parencite{lev2020}.
In addition, accessibility is inherently tied not only to location
but also time \parencite{jar2018},
meaning every place in every time has a level of access
in relation to every other place.

% hard to visualise too
Because of its composite and dynamic nature,
constructing representative cartographic visualizations of accessibility is difficult.
Accessibility visualizations are often constrained
to displaying access in relation to
a limited number of pre-selected places (for example \textcite{wei2018}),
or composing an accessibility index
that can be calculated and mapped for all locations,
in relation to potentially many different things and locations
(for example \textcite{kim2019}).
However, more complex accessibility measures tend to lead to
less usable presentations \parencite{te2014},
while mapping more simple measures could lead to an influx of variations to present.
For example, separating different travel modes, times of day or target locations
would multiply the amount of visualizations needed to present accessibility.

% Not only methodological but also a cartographical challenge 
% something that is not a physical feature nor an easily explainable variable
% ----------------------------------------
% general stuff about cartography - what types of phenomena / maps
% how is accessibility different?
% ----------------------------------------
% Why is visualising access important?
% This could be before the preceding chapter too?
% Leave something for background chapter too

% how interactivity benefits accessibility visualisation (because of how accesibility is)
Even if visualizing every aspect of accessibility
from everywhere to everywhere is impossible,
careful consideration and implementation
of the cartographic methods we have at our disposal might get us closer.
Interactive cartography could be one approach with great potential.
After all, a key principle of interactivity in map presentations is
the map user's ability to change the content of the map \parencite{rot2013b}.
For visualizing accessibility this could mean, for example,
interactive selection of location or travel mode instead of
a static accessibility index that, to a varying extent,
tries to account for everything.
Scientific literature on the topic seems to support the idea of
interactivity in accessibility visualization,
even indicating a need for such presentations.
In a study focusing on the usability of
different accessibility instruments and visualizations,  % accessibility instrument
\textcite{te2014} highlight the importance of
real-time interactivity between the map and the map user.
\textcite{but2018} state that for maps to efficiently communicate accessibility,
they should be as flexible and dynamic as possible.
However, \textcite{but2018} also note that
these qualities are often missing in accessibility visualizations.
Along the same lines,
\textcite{paj2021} find modern accessibility visualizations often complicated,
and lacking in interactivity and flexibility.

% research questions
% Communicating accessibility instead of analyzing it
% How to utilize interactive mapping in communicating accessibility

\subsection{Goals of the work}

In this thesis I focus on
how cartographic interaction can be utilized in presenting accessibility.
My research has two main goals.
The first is to produce an interactive accessibility visualization: %
I develop a system for interactively presenting a massive accessibility dataset,
the \acrlong{ttm} (\acrshort{ttm}) \parencite{fin2023},
on the web.  % TODO introduce the matrix somewhere, maybe here?
The \acrshort{ttm} is a dataset containing travel times for
over a 175 million unique routes, by a multitude of travel modes,
in the Helsinki region, southern Finland.
The sheer size and detail of the dataset allows for endless options
in its presentation, and, from the context of this thesis,
a real opportunity and a need to take advantage of interactivity in visualizing it.
By using cartographic interaction
I aim to extend the multi-location and multi-modal nature of the dataset
to its presentation:
The map user should be able to explore, on a map,
travel times to all locations included in the dataset,
by all travel modes.
However, interaction alone does not yet enable this.
Considering the number of different combinations of location and travel mode,
interaction with the map must happen in real-time,
i.e. with instant visual feedback.
Only with this quality is it possible for the map user to
explore any meaningful portion of the \acrshort{ttm}
in a reasonable amount of time.

When describing the presentation I use the word system.
This is because crafting such a presentation
requires the mapmaker to make much more than a map --
It means designing, implementing and optimizing
a whole that consists of multiple components.
These components all have their distinct purposes yet must function together
to meet the requirements placed upon the whole.
While being a map first and foremost, this particular presentation
requires implementing all the following components:
\begin{itemize}
	\item Preprocessing the dataset (the \acrshort{ttm}) to a mappable format,
	and simplifying it to enable its interactive exploration over the internet
	\item Storing and serving the data, i.e. the backend
	\item Visualizing the data in a web-browser, i.e. the frontend
\end{itemize}

Maps as well as the process of making them are ultimately qualitative
\parencite{cop2009}.
Thus, the overarching topic of this thesis
cannot be fully grasped with only the technical implementation
I laid out as my first goal.
In an attempt to better understand the perspective of the map user,
I focus my second research goal on the human aspect of map use.
Utilizing my interactive accessibility map,
I aim to find out how
people use and understand an interactive presentation of accessibility.
To reach this goal I combine
the interactive map presentation with a survey.


\subsection{Research questions}

My research questions reflect the goals of this study.  % TODO
The first goal, being the interactive map presentation,
necessitates research on the options and tools
needed in its implementation.
The way I approach this research is through qualities
that the resulting system should have.
As real-time interactivity is a priority of the map presentation,
the performance of the system,
i.e. minimizing the latency between
the software receiving input and displaying results of said input,
is one key aspect to consider.
Just as essential is the visual quality of the map
and the intuitiveness of its use.
% discussion perhaps?
I also stress the importance of
\textit{how} the system is designed and developed.
If the software cannot be understood or maintained,
it has little real value,
no matter how performant the implementation or how elegant its user interface.

% With these qualities and the aforementioned components of the system in mind,
% I structrure the research.

% While the resulting system, as a whole, should have all these qualities,
% each part of the system is ultimately its own entity.
% This is important to keep in mind when forming research questions,
% as some qualities are naturally more or less important depending on the component.
% For example,
% visual quality has little to do with backend implementation.

Considering the aforementioned components
and the desired qualities of the system,
my research questions in regard to its implementation are:

\begin{itemize}
	% \item What are the performance bottlenecks of web-based interactive cartography? 
	\item How effective are different data preprocessing approaches at
	improving the performance of the system?
	\item What data storage and serving approach provides
	the most efficient backend for the system?
	\item What differences are there between web mapping libraries
	when visualizing data that rapidly changes based on user interaction?
\end{itemize}  % TODO maybe list approaches -> which is best? Geometry simplification etc...

I answer the first question by evaluating approaches based on three criteria:
\begin{enumerate}
	\item the impact on file size,
	and thus on the time needed to transfer the files over the internet
	\item the impact on rendering speed of the map when mapping the data
	\item the impact on the visual detail of data,
	i.e. how much information is lost
\end{enumerate}

% File sizes can be compared quantitatively.
% When evaluating rendering speed,
% I combine quantitative tests of rendering speed with
% my own qualitative assessment of the rendering performance when using the map.
% I evaluate the visual detail by my own qualitative assessment.

I choose the most efficient solution for
storing and serving the data based on a qualitative assessment of
the available options. I define efficient as:
\begin{enumerate}
	\item Introducing the least possible performance overhead
	\item Introducing the least possible technical complexity
\end{enumerate}

I base my comparison of web mapping libraries on three criteria:
\begin{enumerate}
	\item Visual quality of the map
	\item Responsiveness of the map
	\item User interface capabilities for interacting with the map
\end{enumerate}

% The first two criteria I evaluate qualitatively. The third criteria, while an inherent quality of a mapping-library, Is that too ultimately my own view.  % TODO

To gain insights on how people use the map I ask:  % TODO

\begin{itemize}
	\item What types of map interaction do people prefer
	in different types of usage scenarios?
	% \item Does map interaction change the map user's perception and understanding of the concept of accessibility? If it does, how?
	\item Does map interaction change
	the map user's perception of accessibility? If it does, how?
\end{itemize}

I answer these questions with a survey.
I carry out the survey using a semi-structured online questionnaire
that is structured to:
\begin{enumerate}
	\item Prompt the participant to use the map for different tasks
	\item Ask questions about the participant's experience
	on using the map for completing the tasks
\end{enumerate}

In addition,
the survey has an important role in validating the map presentation.
It provides information on whether the map accomplishes its most important purpose;
to convey information.
In this sense,
the survey results also bring additional weight to
the results of the more technical research questions.
The implementation can be judged by
how well it works when used by real people.

% and whether map interaction impacts their understanding of the mapped phenomenon.

% their order of importance depends on the part of the system being considered.
% I value them differently depending on the  system With these qualities in mind I can then be applied to 
% the parts of the system
% that are visible to the user, i.e. the map application, or frontend.
% Thus, the research questions I ask are:
% at all stages of the implementation.
% In regard to the performance of the map application,
% there are three main components stages at which implementation decisions must be made:  % TODO

% It must also be noted that while important,
% software performance is not the only thing to consider.


% What are the bottlenecks? What type of simplification is most effective?
% What are the pros and cons of different approaches and levels of simplification?

% qs: does the map work? Pros and cons?
% To answer this question, I will divide
% \begin{itemize}
% 	\item Is isochronal representation of accessibility sufficient for real-time web-based map interaction? What are its pros and cons?
% 	\item What kind of application design should be employed to serve and render geographical data fast enough to enable instantaneous interaction?
% \end{itemize}

% \subsection{Outputs}

% \begin{itemize}
% 	\item The map application
% 	\item The bottlenecks of 
% 	\item Does map interaction change the map user's perception and understanding of the concept of accessibility? If it does, how?
% \end{itemize}

% where participants, after being prompted to complete a task with the interactive map,  % TODO

% All of these questions call for a review of relevant literature,
% and of existing accessibility presentations.
% In addition, to better explore especially the third question in practice,
% an interactive web-based accessibility presentation will be implemented.
% This presentation aims to visualize the Helsinki Region Travel Time Matrices \parencite{ten2020}.
% The production of this presentation is carried out through
% an iterative development process with interviews (or questionnaries?)
% to keep in touch with the map user's perspective
% and the ultimately qualitative nature of mapmaking.

% TODO open source good: \parencite{un2020}
% Currently, our data is often acquired, stored
% and used for a single purpose within pillars or
% functions. Access is often difficult, partly
% because of a lack of awareness the data could
% help others or reticence to share what we can.
% \parencite{kra2021} 4.13 open access
