\section{Introduction}
% Start with cartography instead? (kra2021)

% Cartography
% importance: recent developments, data, revealing patterns. See also \textcite{kra2021}.

% Accessibility shortly
% Accessibility from a cartographical angle
% challenges in presenting

Maps are ubiquitous and essential to human civilisation.
In addition to depicting our geographical surroundings,
they help us reveal and understand new patterns and phenomena,
and to gain insights from information
that would otherwise be hard or impossible to interpret \parencite{mac2004}.
Currently we live in a world that is in a state of rapid and
alarming change \parencite{un2023},
but where information is available in an unprecedented abundance.
According to \textcite{un2020}, efficiently harnessing this data
is a requirement for the well-being of the entire planet,
which is why it is acknowledged as
one of the most important issues of the ongoing decade.
As the quantity of data grows
and understanding it becomes more and more demanding,
the importance of maps and cartography is only increasing.
Research on maps, mapping technologies, and cartography is essential,
as it can provide unique ways to synthesise and make sense of complex data,
of which there is no shortage \parencite{kra2021}.
% To summarize, any means that allow us to make sense of an overflow of information are vital,
% and maps and cartography certainly are such means.
% and it is no coincidence that \textcite{kra2021}

% TODO? Cartography in change
% new methods and new topics unlocked
% Much like the world it depicts, cartography itself is changing too.


% This thesis applies all this to accessibility
In this thesis I relate cartography to accessiblity.
When referring to accessibility,
I more specifically mean spatial accessibility, that is,
the ease of reaching valued destinations in physical space \parencite{lev2020}.
Understanding accessibility is important,
as the phenomenon is deeply interlinked in our society.
Accessibility affects the choices and experience of individuals \parencite{kwa1998, kwa2003}
as well as the decision making processes of much larger entities
such as cities or national governments (cite).
It is also directly linked to significant and ongoing issues such as
urban well-being \parencite{zha2011},
equity / justice (cite)
and sustainable transport (cite).
In general, research in accessiblity helps us understand
how societal systems and phenomena function in different
scales and timeframes, and from different points of view.

However, accessibility is an intricate topic to comprehend and analyse.
While the general starting point for defining the term is, quite concisely,
\enquote{potential of opportunities for interaction} \parencite{han1959},  % define interaction in more detail?
more precise definitions and measurements can be constructed in countless ways
depending on the factors deemed impactful to that potential
\parencite{pap2016}.
For example, even with a simple consideration of travel distance as a measure of access,
including all the different things this access can be measured in relation to
would make the amount of variations, i.e. different \enquote{accessibilities},
to measure grow exponentially \parencite{lev2020}.
In addition, accessibility is inherently tied not only to location
but also time \parencite{jar2018},
meaning every place in every time has a level of access
in relation to every other place \parencite{lev2020}.

% hard to visualise too
Because of the composite and dynamic nature of the phenomenon,
constructing representative cartographic visualisations of accessibility is difficult.
Accessibility visualisations are often constrained
to displaying access in relation to
a limited number of pre-selected places (for example \textcite{wei2018}),
or composing an accessibility index that can be calculated and mapped for all locations,
in relation to potentially many different things and locations (for example \textcite{kim2019}).
However, more complex accessibility measures tend to lead to
less usable presentations \parencite{te2014},
while mapping more simple measures could lead to an influx of variations to present.
For example separating different travel modes, times of day or target locations
would multiply the amount of visualisations needed to present accessibility.

% Not only methodological but also a cartographical challenge 
% something that is not a physical feature nor an easily explainable variable
% ----------------------------------------
% general stuff about cartography - what types of phenomena / maps
% how is accessibility different?
% ----------------------------------------
% Why is visualising access important?
% This could be before the preceding chapter too?
% Leave something for background chapter too

% how interactivity benefits accessibility visualisation (because of how accesibility is)
Even if visualising every aspect of accessibility
from everywhere to everywhere is impossible,
careful consideration and implementation
of the cartographic methods we have at our diposal might get us closer.
Interactive cartography could be one approach with great potential.
After all, a key principle of interactivity in map presentations is
the map user's ability to change the content of the map \parencite{rot2013b}.
For visualising accessibility this could mean, for example,
interactive selection of location or travel mode instead of
a static accessibility index that, to a varying extent,
tries to account for everything.
Scientific literature on the topic seems to support the idea of
interactivity in accessibility visualisation,
even indicating a need for such presentations.
In a study focusing on the usablity of
different accessibility instruments and visualisations,  % accessibility instrument
\textcite{te2014} highlight the importance of
real-time interactivity between the map and the map user.
\textcite{but2018} state that for maps to efficiently communicate accessibility,
they should be as flexible and dynamic as possible.
However, \textcite{but2018} also note that
these qualities are often missing in accessibility visualisations.
Along the same lines, \textcite{paj2021} find modern accessibility visualisations often complicated,
and lacking in interactivity and flexibility.

% research questions
% Communicating accessibility instead of analyzing it
% How to utilize interactive mapping in communicating accessibility
This thesis aims to study how accessibility can be presented,
especially considering the methods of interactive cartography.
There are three main research questions.

\begin{enumerate}
	\item What kind of a phenomenon is spatial accessibility representationally?
	What unique aspects does it have?
	\item What kind of cartographic methods and presentations can, or should, 
	be applied when presenting accessibility?
	\item What is the role and potential of interactivity in accessibility visualisations?
\end{enumerate}

To answer question one, I will go into detail of what makes accessibility an unique phenomenon,
especially from a cartographical point of view,
and what requirements this places on visual representations of accessibility.
Here I will focus on the general theory of cartography and especially cartographic interaction,
also previewing previous interactive maps and presentations relative to the topic.
The goal here is to develop an interactive web-based presentation of the
Helsinki Region Travel Time Matrix \parencite{ten2020}.
In addition to the insights gained from the previous research questions,
I will utilize an iterative development process with interviews
to keep in touch with the map user's perspective.

% TODO open source good: \parencite{un2020}
% Currently, our data is often acquired, stored
% and used for a single purpose within pillars or
% functions. Access is often difficult, partly
% because of a lack of awareness the data could
% help others or reticence to share what we can.
% \parencite{kra2021} 4.13 open access
