\section{Introduction}
% Start with cartography instead? (kra2021)

% Cartography
% importance: recent developments, data, revealing patterns. See also \textcite{kra2021}.

% Accessibility shortly
% Accessibility from a cartographical angle
% challenges in presenting


% What accessibility?
The term accessibility is central to this thesis.
Alone, the word simply refers to how easy it is to access something,
with all further meaning derived from
additional explanations and the context in which the word is used.
In this thesis, when referring to accessibility,
I more specifically mean spatial accessibility,
that is, the ease of accessing geographical locations in physical space.

% Geographical accessibility as a phenomenon in general, hard to measure
In this geographical sense,
the general starting point for defining accessibility is often
\enquote{potential of opportunities for interaction} \parencite{han1959}.  % define interaction in more detail?
Depending on the factors considered impactful to that potential,
more precise definitions and measurements can be constructed in many different ways
\parencite{pap2016}.
Even if just using travel distance or time as a measure of access,
including all the different things it can be measured in relation to
would make the amount of variations, i.e. different "accessibilities",
to measure grow exponentially \parencite{lev2020}.
In addition, accessibility is inherently tied not only to location
but also time \parencite{jar2018},
meaning every place in every time has a level of access
in relation to every other place \parencite{lev2020}.
All this makes measuring accessibility in a holistic way a complicated task.

% hard to visualise too
Similarily to defining or measuring it, visualising accessibility is complex.
Accessibility visualisations are often constrained to displaying access in relation to
a limited number of pre-selected places (for example \textcite{wei2018}),
or composing an accessibility index that can be calculated and mapped for all locations,
in relation to potentially many different things and locations (for example \textcite{kim2019}).
However, more complex accessibility measures tend to lead to
less usable presentations \parencite{te2014},
while mapping more simple measures could lead to an influx of variations to present.
For example separating different travel modes, times of day or target locations
would multiply the amount of visualisations needed to present accessibility.

% Not only methodological but also a cartographical challenge 
% something that is not a physical feature nor an easily explainable variable
% ----------------------------------------
% general stuff about cartography - what types of phenomena / maps
% how is accessibility different?
% ----------------------------------------
% Why is visualising access important?
% This could be before the preceding chapter too?
% Leave something for background chapter too

% how interactivity benefits accessibility visualisation (because of how accesibility is)
Even if visualising every aspect of accessibility
from everywhere to everywhere is impossible,
interactive visualisations could offer some benefits for presenting accessibility.
After all, a key principle of interactivity in map presentations is
the map user's ability to change the content of the map \parencite{rot2013b}.
For visualising accessibility this could mean, for example,
interactive selection of location or travel mode instead of
a static accessibility index that, to a varying extent,
tries to account for everything.
Scientific literature on the topic seems to support the idea of
interactivity in accessibility visualisation,
even indicating a need for such presentations.
In a study focusing on the usablity of
different accessibility instruments and visualisations,  % accessibility instrument
\textcite{te2014} highlight the importance of
real-time interactivity between the map and the map user.
\textcite{but2018} state that for maps to efficiently communicate accessibility,
they should be as flexible and dynamic as possible.
However, \textcite{but2018} also note that
these qualities are often missing in accessibility visualisations.
Along the same lines, \textcite{paj2021} find modern accessibility visualisations often complicated,
and lacking in interactivity and flexibility.

% research questions
% Communicating accessibility instead of analyzing it
% How to utilize interactive mapping in communicating accessibility
This thesis approaches accessibility from a cartographical angle,
especially in the context of interactive mapping.
There are three main research questions.
\begin{enumerate}
	\item What kind of a phenomenon is spatial accessibility representationally?
	What unique aspects does it have?
	\item Which cartographic methods and presentations can, or should, 
	be applied when presenting accessibility?
	\item What is the role and potential of interactivity in accessibility visualisations?
\end{enumerate}
% To answer question one, I will go into detail of what makes accessibility an unique phenomenon,
% especially from a cartographical point of view,
% and what requirements this places on visual representations of accessibility.
% Here I will focus on the general theory of cartography and especially cartographic interaction,
% also previewing previous interactive maps and presentations relative to the topic.
% The goal here is to develop an interactive web-based presentation of the
% Helsinki Region Travel Time Matrix \parencite{ten2020}.
% In addition to the insights gained from the previous research questions,
% I will utilize an iterative development process with interviews
% to keep in touch with the map user's perspective.
