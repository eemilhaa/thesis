\section{Introduction}
% Start with cartography instead? (kra2021)

% Cartography
% importance: recent developments, data, revealing patterns. See also \textcite{kra2021}.

% Accessibility shortly
% Accessibility from a cartographical angle
% challenges in presenting

Maps are ubiquitous and essential to human civilization.
In addition to depicting our geographical surroundings,
they help us reveal and understand new patterns and phenomena,
and to gain insights from information
that would otherwise be hard or impossible to interpret \parencite{mac2004}.
Currently, we live in a world that is in a state of rapid and
alarming change \parencite{un2023},
but where information is available in an unprecedented abundance.
Acknowledged as one of the most pressing issues of the ongoing decade,
efficiently harnessing this data is a requirement
for the well-being of the entire planet \parencite{un2020}.
As the quantity of data grows
and understanding it becomes more and more demanding,
the relevance of maps and cartography is only increasing.
Maps, mapping technologies, and cartography are essential,
as they can provide unique ways to synthesize and make sense of complex data,
of which there is no shortage \parencite{kra2021}.
% To summarize, any means that allow us to make sense of an overflow of information are vital,
% and maps and cartography certainly are such means.
% and it is no coincidence that \textcite{kra2021}

% TODO? Cartography in change
% new methods and new topics unlocked
% Much like the world it depicts, cartography itself is changing too.


% This thesis applies all this to accessibility
In this thesis I relate cartography to accessibility.
When referring to accessibility,
I more specifically mean spatial accessibility, that is,
the ease of reaching valued destinations in physical space \parencite{lev2020}.
Understanding accessibility is important,
as the phenomenon is deeply interlinked in our society.
Accessibility affects the choices and experiences of individuals \parencite{kwa1998, kwa2003}
as well as the planning and governance processes
of much larger entities such as cities \parencite{cur2010, low2015}.
It is also directly linked to significant and ongoing issues such as
urban well-being \parencite{zha2011},
equity and justice \parencite{per2017, che2020}
and sustainable transport and mobility \parencite{son2017, mah2019}.
In general, research in accessibility helps us understand
how societal systems and phenomena function
in different scales and timeframes, and from different points of view.

% difficulty in defining and measuring access
However, analysing or even comprehending accessibility
in a holistic manner is not simple.
While a common starting point for defining the term is, quite concisely,
\enquote{potential of opportunities for interaction} \parencite{han1959},  % define interaction in more detail?
more precise definitions and measurements can be constructed in countless ways
depending on the factors deemed impactful to that potential
\parencite{pap2016}.
For example, even with a simple consideration of travel distance as a measure of access,
including all the different things this access can be measured in relation to
would make the amount of variations, i.e. different \enquote{accessibilities},
to measure grow exponentially \parencite{lev2020}.
In addition, accessibility is inherently tied not only to location
but also time \parencite{jar2018},
meaning every place in every time has a level of access
in relation to every other place.

% hard to visualise too
Because of its composite and dynamic nature,
constructing representative cartographic visualizations of accessibility is difficult.
Accessibility visualizations are often constrained
to displaying access in relation to
a limited number of pre-selected places (for example \textcite{wei2018}),
or composing an accessibility index
that can be calculated and mapped for all locations,
in relation to potentially many different things and locations
(for example \textcite{kim2019}).
However, more complex accessibility measures tend to lead to
less usable presentations \parencite{te2014},
while mapping more simple measures could lead to an influx of variations to present.
For example, separating different travel modes, times of day or target locations
would multiply the amount of visualizations needed to present accessibility.

% Not only methodological but also a cartographical challenge 
% something that is not a physical feature nor an easily explainable variable
% ----------------------------------------
% general stuff about cartography - what types of phenomena / maps
% how is accessibility different?
% ----------------------------------------
% Why is visualising access important?
% This could be before the preceding chapter too?
% Leave something for background chapter too

% how interactivity benefits accessibility visualisation (because of how accesibility is)
Even if visualizing every aspect of accessibility
from everywhere to everywhere is impossible,
careful consideration and implementation
of the cartographic methods we have at our disposal might get us closer.
Interactive cartography could be one approach with great potential.
After all, a key principle of interactivity in map presentations is
the map user's ability to change the content of the map \parencite{rot2013b}.
For visualizing accessibility this could mean, for example,
interactive selection of location or travel mode instead of
a static accessibility index that, to a varying extent,
tries to account for everything.
Scientific literature on the topic seems to support the idea of
interactivity in accessibility visualization,
even indicating a need for such presentations.
In a study focusing on the usability of
different accessibility instruments and visualizations,  % accessibility instrument
\textcite{te2014} highlight the importance of
real-time interactivity between the map and the map user.
\textcite{but2018} state that for maps to efficiently communicate accessibility,
they should be as flexible and dynamic as possible.
However, \textcite{but2018} also note that
these qualities are often missing in accessibility visualizations.
Along the same lines, \textcite{paj2021} find modern accessibility visualizations often complicated,
and lacking in interactivity and flexibility.

% research questions
% Communicating accessibility instead of analyzing it
% How to utilize interactive mapping in communicating accessibility

In this thesis I focus on
how cartographic interaction can be utilized in presenting accessibility.
My research has two main goals.
The first is to produce an interactive accessibility visualization -- %
a system for interactively presenting a massive accessibility dataset,
the Helsinki region travel time matrix \parencite{fin2023},
on the web.  % TODO introduce the matrix somewhere, maybe here?
Interactivity in this context means that
the map user should be able to
explore the whole dataset in real-time
and with as little restrictions as possible.
When describing the presentation I use the word system.
This is because crafting such a presentation
requires the mapmaker to make much more than a map --
It means designing, implementing and optimizing
many different pieces of software that must all function together
to meet the requirements real-time interactivity entails.

Maps as well as the process of making them are ultimately qualitative
\parencite{cop2009}.
As the overarching topic of this thesis
cannot be fully grasped with only the technical implementation
I laid out as my first research goal,
I concentrate my second goal on the human aspect of map use.
Utilizing my interactive accessibility map,
I aim to find out how
people use and understand an interactive presentation of accessibility.
To reach this goal I combine
the interactive map presentation with a survey.
I carry out the survey using a semi-structured online questionnaire
that asks the map user about how they use the map for different tasks.

My research questions reflect the structure of the goals of this study.  % TODO
The research questions that must be answered to
enable this particular map presentation,
and to reach my first goal, are:
The research questions I aim to answer are:
\begin{enumerate}
	\item How to simplify a massive accessibility dataset to be explorable in real-time over the internet?
\end{enumerate}

I further divide this question into two sub questions:

% What are the bottlenecks? What type of simplification is most effective?
% What are the pros and cons of different approaches and levels of simplification?
To answer this question, I will divide
\begin{enumerate}
	\item Is isochronal representation of accessibility sufficient for real-time web-based map interaction? What are its pros and cons?
	\item What kind of application design should be employed to serve and render geographical data fast enough to enable instantaneous interaction?
\end{enumerate}

\begin{enumerate}
	\item What types of map interaction do people use and prefer in different types of usage scenarios?
	\item Does map interaction change the map user's perception and understanding of the concept of accessibility? If it does, how?
\end{enumerate}

Additionally, the questionnaire allow

% where participants, after being prompted to complete a task with the interactive map,  % TODO

% All of these questions call for a review of relevant literature,
% and of existing accessibility presentations.
% In addition, to better explore especially the third question in practice,
% an interactive web-based accessibility presentation will be implemented.
% This presentation aims to visualize the Helsinki Region Travel Time Matrices \parencite{ten2020}.
% The production of this presentation is carried out through
% an iterative development process with interviews (or questionnaries?)
% to keep in touch with the map user's perspective
% and the ultimately qualitative nature of mapmaking.

% TODO open source good: \parencite{un2020}
% Currently, our data is often acquired, stored
% and used for a single purpose within pillars or
% functions. Access is often difficult, partly
% because of a lack of awareness the data could
% help others or reticence to share what we can.
% \parencite{kra2021} 4.13 open access
