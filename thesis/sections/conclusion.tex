\section{Conclusion}

Cartographic interaction allows us to better make sense of a world overflowing with information.
Studying how cartographic interaction and map interfaces work
as well as assessing the tools of the modern-day interactive cartographer
is necessary to cartographic research.

This study showed that, while possible, achieving a responsive web-based interface to complex spatial data
necessitates tradeoffs as well as careful consideration of the technologies used in its implementation.
From a user perspective, a real-time interactive map was a functional interface,
enabling a genuinely different way of exploring and assessing massive spatial data on a map.
% and a dynamic approach to manipulating a map affected its interpretation.
At a general level,
this study can be seen as a reproducible approach to assessing web-mapping tools
and testing a map interface by bringing it to real-world users as easy-to-access,
working, software.

The focus in this thesis was on a very specific kind of human-map interaction,
as enabled by a specific web map application.
In addition to these perspectives,
further study regarding different types of map interaction, map interfaces as well as map users is key
to advancing the agenda within interactive cartography.
% Just as important is to keep up with the progress of mapping technologies.
% Cartographic interaction cannot be 
In addition to scholars,
cartographic interaction and interactive maps extend the relevance of cartography to
anyone using or developing map applications.
% This goes both ways, as 
While cartography has often been connected to the search for
an optimal representation of the world,
cartographic interaction
has the unique potential to empower the map user to make that representation for themselves.

% Whether the positives of any given approach to cartographic interaction outweigh its negatives
% is a question as important as pinpointing what those positives and negatives are in the first place.

% These results should be taken in the context of map interfaces where interaction requires
% frequent re-fetching and re-rendering of data.

% [Findings] Dramatic alterations in the egg-laying behaviour of Abc xyz beetles were evident in our study. Moreover, the impacts on beetle behaviour were affected to a greater extent by white light than by yellow light.

% [Limitations, Scope for further research] Further work is needed to clarify the role of light pollution in disrupting other behaviours in these beetles, as well as in other local insects. Considering the general move from traditional yellow lighting to white LEDs, outdoor lighting will need to be modified to minimise the detrimental effects on insect populations.

% [Strong concluding sentence] The spread of urbanisation cannot be curbed, but appropriate steps can be undertaken to minimise disruptions to biological rhythms of local fauna.

% Based on my findings,
% the geometrical complexity of data can be seen as a
% major limiting factor to the responsiveness of web map interfaces.
% This should be taken in the context of interfaces where interaction requires
% frequent re-rendering of data.
% I observed notable differences between web mapping libraries,
% suggesting that technological choices play a significant role
% in the capabilities of map interfaces.
% At a general level,
% my findings and the development process show that a
% technological perspective into cartographic interaction
% is necessary




% Map functionalities cannot be afterthoughts.
