\section{Conclusions}

% Cartographic interaction
This study focused on cartographic interaction from two points of view.
From a technology-centric perspective,
I implemented a highly interactive web map application to
find which factors limit map responsiveness the most,
and to assess the software libraries with which cartographic interaction is crafted on the web.
From a user-centric perspective, I carried out a survey to
study how such an interface is used,
and how map users perceive the mapped phenomenon
through dynamic real-time cartographic interaction.

Based on my findings,
the geometrical complexity of data can be seen as a
major limiting factor to the responsiveness of web map interfaces.
This should be taken in the context of interfaces where interaction requires
frequent re-rendering of data.
I observed notable differences between web mapping libraries,
suggesting that technological choices play a significant role
in the capabilities of map interfaces.
At a general level,
my findings and the development process show that a
technological perspective into cartographic interaction
is necessary

As web map applications grow both in complexity and in demands for performance and usability,
keeping up with the progress of web technologies is essential.

Map functionalities cannot be afterthoughts.

Cartographic interaction as enabled by digital map interfaces
