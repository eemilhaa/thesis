\section{Discussion}

\subsection{}
% To summarize, any means that allow us to make sense of an overflow of information are vital,
% and maps and cartography certainly are such means.
% and it is no coincidence that \textcite{kra2021}

\subsection{Reproducibility}
A clear was identified that there is a clear lack of tools
specifically designed for modelling active mobility that are open source,
include interactive scenario building,
and can easily be transferred to new study areas \parencite{paj2021}.

\subsection{}
Discussion and conclusions about the results

Flexibility in the map? Tradeoffs - now priority is on instant interaction instead of detailed access maps.
What would the map look like with different decisions at the start? For example detail over speed.
This would mean different evaluation of everything (preprocess, back / frontend)

Green code

Maintainability

Further improvements of the map application:

different things to measure access in relation to (population),
accessibility of different times (matrices from different years),
geometry simplification,
custom basemaps,
more discoverable UI,

Just like the \acrshort{ttm}, every part of the visualisation is open source and reproducible elsewhere

Mapmaking in the age of web mapping:

JavaScript fatigue: The web ecosystem is large and changes constantly

Mapmaking is a very small fraction of all the work that goes into making the map

But everything else must still be made, or the map wont 1. be there for anyone to use 2. have any data to show 3. be there tomorrow when someone must maintain it.

Software development process is not reflected in the structure of a scientific paper
