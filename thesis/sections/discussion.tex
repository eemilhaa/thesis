\section{Discussion}

Some ideas for discussion:
\begin{itemize}
	\item Pros and cons of real-time interaction:
	Implementing something like the hover mode could be worth it in the sense
	that it provides a genuinely different way to explore massive data.
	Depending on the data being visualized,
	the loss of detail to enable such a presentation could be immense,
	potentially leading to a shallow presentation with little insight to offer.
	\item Too much data / too precise data:
	Scaling this type of approach to even more data could be difficult. Can data be too detailed?
	\item Tradeoffs in the map:
	now priority is on instant interaction instead of detailed maps.
	What would the map, and by extension the whole study, look like with different priorities?
	\item Reproducibility and open science:
	The \acrshort{ttm} is not only an open dataset but also an open method.
	This work continues on that track,
	as every component is completely open-source \& openly available.
	\item Amount of work \& the need for cooperation:
	Mapmaking is a very small fraction of all the work that goes into implementing cartographic interaction,
	yet all the other work must be done, or the map will not exist.
	Cooperation with developers, map users and cartographers is key.
\end{itemize}

\section{Conclusions}
