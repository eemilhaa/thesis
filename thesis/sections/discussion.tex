\section{Discussion}

Some ideas for discussion:
\begin{itemize}
	\item Pros and cons of real-time interaction:
	Implementing something like the hovering mode could be worth it in the sense
	that it provides a genuinely different way to explore massive data.
	Depending on the data being visualized,
	the loss of detail to enable such a presentation could be immense,
	potentially leading to a shallow presentation with little insight to offer.
	\item Too much data / too precise data:
	Scaling this type of approach to even more data could be difficult. Can data be too detailed?
	\item Tradeoffs in the map:
	now priority is on instant interaction instead of detailed maps.
	What would the map, and by extension the whole study, look like with different priorities?
	\item Reproducibility and open science:
	The \acrshort{ttm} is not only an open dataset but also an open method.
	This work continues on that track,
	as every component is completely open-source \& openly available.
	\item Amount of work \& the need for cooperation:
	Mapmaking is a very small fraction of all the work that goes into implementing cartographic interaction,
	yet all the other work must be done, or the map will not exist.
	Cooperation with developers, map users and cartographers is key.
\end{itemize}

% My results showed that...
\subsection{Technical considerations and the process of creating cartographic interaction}

a high level of data simplification was needed to enable real-time interaction with the map
in the context of this particular application.



Based on these results
I can also reason about the performance bottlenecks,
but only in the context of this particular application, i.e. rapid real-time interaction.
In this context the bottleneck is definitely rendering speed,
which is most affected by the geometrical complexity.
For example, the map would in no way work if rendering individual grid cells,
even with instant data transfer.

While reliably generalizing these findings based on this study alone is impossible,
I would form the hypothesis that if the file size of a given data consists mostly of geometries,
rendering performance will limit map responsiveness before file sizes become an issue.

Many of the tradeoffs above are not inherent to cartographic interaction,
rather they are technical limitations.

% The map presentation and survey results from technical perspective
Too much data / too precise data:
Scaling this type of approach to even more data could be difficult.
Can data be too detailed?

Amount of work and the need for cooperation:
Mapmaking is a very small fraction of all the work
that goes into implementing cartographic interaction,
yet all the other work must be done,
or the map will not exist.
Cooperation with developers, map users and cartographers is key.

Generalization of results is difficult if not impossible

% https://andrejgajdos.com/leaflet-developer-guide-to-high-performance-map-visualizations-in-react/
Difficulty in assessing options: Testing the technology or the implementation.

Software to support GeoVisual Analytics should be lightweight, easily % and2010 FIXME COPYPASTA
deployable and usable, rather than huge and complex like current GIS

Importance of keeping up with technical change \parencite{rot2014},
modular software to enable keeping up with change.
Newer libraries observed more performant and support what is considered essential out of the box.

% Accessibility? A whole different technology stack and an immense amount of work


\subsection{Real time cartographic interaction and its tradeoffs}

% The map presentation and survey results from cartography perspective
Implementing something like the hovering mode could be worth it in the sense
that it provides a genuinely different way to explore massive data.
Depending on the data being visualized,
the loss of detail to enable such a presentation could be immense,
potentially leading to a shallow presentation with little insight to offer.

Hovering mode was favored regardless of task type → dynamic better?

Overall, the results of the survey indicate that the map worked,
both in the sense of usability and conveying information about the mapped phenomenon.

Tradeoffs in the map:
now priority is on instant interaction instead of detailed maps.

Different perception depending on mode of interaction,
harder to understand hovering mode →
more information reflects this particular phenomenon better?


\subsection{Data access}
% inc2012


% Not complete without representation
In many cases, open access to data does not by itself enable
the exploration or understanding of said data \parencite{obr2016}.
It could be argued that the \acrshort{ttm} is a prime example of this.
In its raw, unprocessed, form,
the dataset constitutes gigabytes of data spread over thousands of plain text files
that need to be further combined with an external statistical grid to enable
geographical representation of the data.
Merely comprehending the structure of the dataset requires expertise,
proving to be anything but obvious even to university students specializing in the field
-- the author included.
Even though the latest iteration of the dataset \parencite{fin2023}
includes an easier-to-map data format with geometries included,
actually mapping the \acrshort{ttm} would still require
desktop \acrshort{gis} software or a programming environment,
coupled with the know-how of using either option.
An open representation for open data

% The representation from the angle of access to understanding data, reproducibility and open science
This work continues on that track,
as every component is completely open-source \& openly available.

\subsection{Open source}
% TTM overview: measurements, modelling, dataset
Not only a dataset, the \acrshort{ttm} is an extensive showcase of open science.
Scalable Kubernetes / OpenShift deployments → Reproducibility at the level of the entire application
Containerization → Reproducibility at the level of the components of the application

\subsection{What was not mapped}

A lot. By this I mean that the options in
conceptualizing, implementing and studying an interactive map presentation are innumerable.

What would the map, and by extension the whole study, look like with different priorities?

With different definitions of accessibility?

Further research:
