\section{Discussion}

Some ideas for discussion:
\begin{itemize}
	\item Pros and cons of real-time interaction:
	Implementing something like the hovering mode could be worth it in the sense
	that it provides a genuinely different way to explore massive data.
	Depending on the data being visualized,
	the loss of detail to enable such a presentation could be immense,
	potentially leading to a shallow presentation with little insight to offer.
	\item Too much data / too precise data:
	Scaling this type of approach to even more data could be difficult. Can data be too detailed?
	\item Tradeoffs in the map:
	now priority is on instant interaction instead of detailed maps.
	What would the map, and by extension the whole study, look like with different priorities?
	\item Reproducibility and open science:
	The \acrshort{ttm} is not only an open dataset but also an open method.
	This work continues on that track,
	as every component is completely open-source \& openly available.
	\item Amount of work \& the need for cooperation:
	Mapmaking is a very small fraction of all the work that goes into implementing cartographic interaction,
	yet all the other work must be done, or the map will not exist.
	Cooperation with developers, map users and cartographers is key.
\end{itemize}

% My results showed that...
\subsection{Technological considerations for cartographic interaction on the web}

With the technical side of this study, my first goal was to pinpoint
the performance bottlenecks of
an interactive web map application.
I also aimed to find what data preprocessing and simplifying approaches
were needed to overcome them.
My assessment of the preprocessing methods showed that, in the case of this particular
map application, the geometrical complexity of the data was the most limiting factor
when striving for real-time cartographic interaction.
While reliably generalizing these findings based on this study alone is impossible,
I would form the following hypothesis:
In a web mapping context, If a given data consists mostly of geometries, i.e.
features that must be drawn and re-drawn on the map in real-time,
rendering performance will limit map responsiveness before file sizes become an issue.
Reflecting this,
the data preprocessing methods that were needed
to enable a responsive map interface in this study
consisted of heavy geometrical simplification of the data.
As a result, much of the detail on the map was lost,
but the real-time exploration of said data was made possible.
This is an important factor in the design of
web-based map interfaces in general,
as such loss of detail, while providing capabilities for fast exploration,
could potentially lead to
a shallow presentation with little insight to offer.
Considering such tradeoffs will only become more relevant moving forwards,
as the amount of data to map is only increasing \parencite{kra2017}.

My second technology-centric goal was to assess
the differences between web mapping libraries
when used as a platform for real-time cartographic interaction.
My results showed that such differences do exist,
and that they do matter from the perspective of the map interface --
both in terms of using and developing it.
An important detail is also that
it was the \textit{newer} web-mapping libraries that I observed most performant,
enabling real-time cartographic interaction.
Also, the newer libraries included features and capabilities
that might today be considered essential,
without the need for plugins or self-made modifications.
Depending on the use case, such things could include GPU acceleration,
vector base maps, or UI framework integration.
In general, the web is often considered the fastest changing
software ecosystem \parencite{mik2019, tai2017} --
a fact also recognized in the web mapping context
\parencite{rot2014, vee2017}.
The aforementioned findings reinforce this, emphasizing the need of the web cartographer
having to keep up with the rapid innovation happening on the platform:
Depending on the technological choices, the map application implemented in this study
either works fluidly, or is simply nonfunctional.

A more general discussion can also be raised on the topic of what exactly is
a \enquote{web cartographer}.
Making maps for the web can range from simple
single-purpose maps to extensive web applications with
client- and server-side components \parencite{vee2017, mai2017}.
In the first case,
the map could be seen as something not too far removed from
what a map is traditionally seen as: for example a communicative
device intended to get its predetermined message across to the map reader
\parencite{kol1969},
or a single representation of the mapped topic \parencite{tyn1992}.
Also, the expertise required to craft such a map is mostly concerned with
the cartographic method, only applied using the relevant technologies \parencite{rot2021}.
In the second case,
the map is simultaneously a UI, a web application, and, of course, a map.
Crafting such a map requires knowledge on many topics:
The cartographer is a UI designer and a usability engineer,
but also increasingly a developer \parencite{rot2017, mai2017}.
My study and the development process I carried out to enable it supports this.
Real-time cartographic interaction was an aspect that had to be considered
in the implementation of all the components of a client-server web application --
mapmaking, in the traditional sense, was a very small portion of the work
needed to make the map.
Based on this,
the web cartographer could be theorized as
someone more akin to the elusive full-stack web developer,
\enquote{covering the responsibility of an entire IT
department} \parencite[p.~370]{tai2021}.
Who- or whatever they might be, conceptualizing the role of the web cartographer is important.
Much like software in general, maps and cartographic interaction
are more and more often made for the web \parencite{tai2017, rot2021, rot2014, vee2017}.

The technical side of this study





Importance of keeping up with technical change \parencite{rot2014},

modular software to enable keeping up with change.

Software to support GeoVisual Analytics should be lightweight, easily % and2010 FIXME COPYPASTA
deployable and usable, rather than huge and complex like current GIS


In general, the technical implementation showed that


% The map presentation and survey results from technical perspective
Too much data / too precise data:
Scaling this type of approach to even more data could be difficult.
Can data be too detailed?

Amount of work and the need for cooperation:
Mapmaking is a very small fraction of all the work
that goes into implementing cartographic interaction,
yet all the other work must be done,
or the map will not exist.
Cooperation with developers, map users and cartographers is key.


Generalization of results is difficult if not impossible

% https://andrejgajdos.com/leaflet-developer-guide-to-high-performance-map-visualizations-in-react/
Difficulty in assessing options: Testing the technology or the implementation.



% Accessibility? A whole different technology stack and an immense amount of work


% \subsection{Real time cartographic interaction and its tradeoffs}
\subsection{An interactive web map as a representation and an interface}

% \item Which types of interaction available in the map do people use
% in different types of usage scenarios?
% \item Does dynamic real-time interaction with the map change  % FIXME new term?
% the map user's perception of the mapped phenomenon? If it does, how?
By carrying out the survey I firstly aimed to find out what types of interaction
map users utilize in different scenarios of map use.
My results showed that, in the context of map use \textit{as prompted in the questionnaire},
the participants preferred as dynamic as possible interaction with the map.
This was true in two different kinds of tasks,
in addition to the participants rating the more dynamic mode of interaction
more useful than the less dynamic one.
This supports the recognized need for dynamic interfaces in exploring and analysing
spatial data \parencite{eds2008, but2018}.
However, generalizing these findings to answer what mode or style of interaction is
\enquote{the best} is not possible here,
nor is it really a realistic or sensible proposition.
At the level of the map interface, analytical thinking as facilitated by cartographic interaction
consists of immensely diverse \enquote{usage scenarios} and \enquote{tasks}
\parencite{rob2017b, and2010},
all with their respective needs for interaction capabilities with the map
\parencite{rot2013a, rot2015}.
This was also reflected in my results:
In addition to singular map functionalities,
participants highlighted different combinations of map interactions as most useful,
and also mentioned how the value of a given type or style of interaction is
dependent on the usage scenario.  % experienced freedom in interface?

My second goal with the survey was to see if, and how,
dynamic real-time cartographic interaction changes the way
map users perceive the mapped phenomenon.
Based on my results, the participants' perception of the mapped phenomenon did change
depending on the mode of interaction they used in examining it.
A key aspect is also that the participants found it harder to decide which one
of two locations is more accessible
when using the more dynamic hovering mode to carry out the assessment.
However, the participants also regarded the map interface as easy to use,
chose to use the hovering mode when given the chance,
and rated the hovering mode as the more useful mode
of interaction with the map in general.
Based on this, it can be reasoned that the increased difficulty of the
comparison is unlikely to be caused by
difficulties in using the mode of interaction,
but rather a by-product of
the amount and speed of information that the user is presented with when using it.
This, combined with the differing results depending on mode of interaction,
shows that a dynamic, real-time, mode of interaction did provide
a genuinely different representation of the mapped phenomenon.
Bringing this into the context of accessibility, i.e. the mapped phenomenon,
it could also be said that a definitive comparison of the accessibilities of
two locations is by no means a simple task --
The difficulty in making such a comparison could reflect the complexity of the phenomenon,
and, should this be the case, signal that dynamic real-time interaction did provide
a more accurate way to represent that complexity to the map user.

 could be argued to allow for a more reflective
representation of the mapped phenomenon


All this also further highlights the value that more dynamic
interaction can introduce to map interfaces.

% Identifying the types of tasks where a given mode of interaction if most used.

% The map presentation and survey results from cartography perspective
Implementing something like the hovering mode could be worth it in the sense
Depending on the data being visualized,
the loss of detail to enable such a presentation could be immense,
potentially leading to a shallow presentation with little insight to offer.

% as dynamic as possible?
Hovering mode was favored regardless of task type → dynamic better?
\parencite{but2018}

Overall, the results of the survey indicate that the map worked,
both in the sense of usability and conveying information about the mapped phenomenon.

Tradeoffs in the map:
now priority is on instant interaction instead of detailed maps.

Different perception depending on mode of interaction,
harder to understand hovering mode →
more information reflects this particular phenomenon better?


% \subsection{The value of web-based cartographic interaction}
\subsection{Open science, data and presentation}
% inc2012
The \acrshort{ttm} is an extensive showcase of open science.


% Not complete without representation
In many cases, open access to data does not by itself enable
the exploration or understanding of said data \parencite{obr2016}.
It could be argued that the \acrshort{ttm} is a prime example of this.
In its raw form,
the dataset consists of gigabytes of plain text spread over thousands of files
that need to be further combined with an external statistical grid to enable
geographical presentation of the data.
Merely comprehending the structure of the dataset requires expertise,
proving to be anything but obvious even to university students specializing in the field
-- the author included.
Even though the latest iteration of the dataset \parencite{fin2023}
includes an easier-to-map data format with geometries included,
actually mapping the \acrshort{ttm} would still require
desktop \acrshort{gis} software or a programming environment,
coupled with the know-how of using either option.
An open representation for open data

% The representation from the angle of access to understanding data, reproducibility and open science
This work continues on that track,
as every component is completely open-source \& openly available.

% TTM overview: measurements, modelling, dataset
Scalable Kubernetes / OpenShift deployments → Reproducibility at the level of the entire application
Containerization → Reproducibility at the level of the components of the application

\subsection{What was not mapped}

A lot. By this I mean that the options in
conceptualizing, implementing and studying an interactive map presentation are innumerable.

What would the map, and by extension the whole study, look like with different priorities?

With different definitions of accessibility?

Sustainability of interaction?

Further research:
This highlights the fact that the technical tradeoffs and limitations found in this study
are in no way inherent to cartographic interaction:
Rather, they are technical limitations, that change in time.
