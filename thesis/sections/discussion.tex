\section{Discussion}

% 	Pros and cons of real-time interaction:
% 	Implementing something like the hovering mode could be worth it in the sense
% 	that it provides a genuinely different way to explore massive data.
% 	Depending on the data being visualized,
% 	the loss of detail to enable such a presentation could be immense,
% 	potentially leading to a shallow presentation with little insight to offer.

% 	Too much data / too precise data:
% 	Scaling this type of approach to even more data could be difficult. Can data be too detailed?

% 	Tradeoffs in the map:
% 	now priority is on instant interaction instead of detailed maps.
% 	What would the map, and by extension the whole study, look like with different priorities?

% 	Reproducibility and open science:
% 	The \acrshort{ttm} is not only an open dataset but also an open method.
% 	This work continues on that track,
% 	as every component is completely open-source \& openly available.

% 	Amount of work \& the need for cooperation:
% 	Mapmaking is a very small fraction of all the work that goes into implementing cartographic interaction,
% 	yet all the other work must be done, or the map will not exist.
% 	Cooperation with developers, map users and cartographers is key.

% My results showed that...
\subsection{Technological considerations for cartographic interaction on the web}

With the technical side of this study, my first goal was to pinpoint
the performance bottlenecks of
an interactive web map application.
I also aimed to find what data preprocessing and simplifying approaches
were needed to overcome them.
My assessment of the preprocessing methods showed that, in the case of this particular
map application, the geometrical complexity of the data was the most limiting factor
when striving for real-time cartographic interaction.
While reliably generalizing these findings based on this study alone is impossible,
I would form the following hypothesis:
In a web mapping context, If a given data consists mostly of geometries, i.e.
features that must be drawn and re-drawn on the map in real-time,
rendering performance will limit map responsiveness before file sizes become an issue.
Reflecting this,
the data preprocessing methods that were needed
to enable a responsive map interface in this study
consisted of heavy geometrical simplification of the data.
As a result, much of the detail on the map was lost,
but the real-time exploration of said data was made possible.
This is an important factor in the design of
web-based map interfaces in general,
as such loss of detail, while providing capabilities for fast exploration,
could potentially lead to
a shallow presentation with little insight to offer.
Considering such tradeoffs will only become more relevant moving forwards,
as the amount of data to map is only increasing \parencite{kra2021, un2023}.

My second technology-centric goal was to assess
the differences between web mapping libraries
when used as a platform for real-time cartographic interaction.
My results showed that such differences do exist,
and that they do matter from the perspective of the map interface --
both in terms of using and developing it.
An important detail is also that
it was the \textit{newer} web-mapping libraries that I observed most performant,
enabling real-time cartographic interaction.
Also, the newer libraries included features and capabilities
that might today be considered essential,
without the need for plugins or self-made modifications.
Depending on the use case, such things could include GPU acceleration,
vector base maps, or UI framework integration.
In general, the web is often considered the fastest changing
software ecosystem \parencite{mik2019, tai2017} --
a fact also recognized in the web mapping context
\parencite{rot2014, vee2017}.
The aforementioned findings reinforce this, emphasizing the need of the web cartographer
having to keep up with the rapid innovation happening on the platform:
Depending on the technological choices, the map application implemented in this study
either works fluidly, or is simply nonfunctional.

A more general discussion can also be raised on the topic of what exactly is
a \enquote{web cartographer}.
Making maps for the web can range from simple
single-purpose maps to extensive web applications with
client- and server-side components \parencite{vee2017, mai2017}.
In the first case,
the map could be seen as something not too far removed from
what a map is traditionally seen as: for example a communicative
device intended to get its predetermined message across to the map reader
\parencite{kol1969},
or a single representation of the mapped topic \parencite{tyn1992}.
Also, the expertise required to craft such a map is mostly concerned with
the cartographic method, only applied using the relevant technologies \parencite{rot2021}.
In the second case,
the map is simultaneously a UI, a web application, and, of course, a map.
Crafting such a map requires knowledge on many topics:
The cartographer is a UI designer and a usability engineer,
but also increasingly a developer \parencite{rot2017, mai2017}.
My study, and the development process I carried out to enable it, supports this.
Real-time cartographic interaction was an aspect that had to be considered
in the implementation of all the components of a client-server web application --
mapmaking, in the traditional sense, was a very small portion of the work
needed to make the map.
Based on this,
the web cartographer could be theorized as
someone more akin to the elusive full-stack web developer,
\enquote{covering the responsibility of an entire IT
department} \parencite[p.~370]{tai2021}.
Who- or whatever they might be, conceptualizing the role of the web cartographer is important.
Much like software in general, maps and cartographic interaction
are more and more often made for the web \parencite{tai2017, rot2021, vee2017}.

The main limitation of the technical side of this study
was the lack of quantitative data to back up
the claims related to the responsiveness of the map interface.
Another limitation is that testing software libraries by
actively developing something with them introduces
a level of uncertainty to the results:
The correctness of a given computer program is difficult to prove
\parencite{dij1970, amm2016},
which could mean the results are more related to the particular program
instead of the library it was made with.
Minimizing this uncertainty necessitated
a highly modular design of the client-side application.
One output of this study is the notion that such design did allow for
the relatively rapid and efficient testing of different web mapping libraries,
providing a concrete example of
how the recognized need of keeping up with evolving web mapping tools \parencite{rot2014, rot2021}
could be satisfied at the software level.


% Cooperation with developers, map users and cartographers is key.


% Generalization of results is difficult if not impossible



% \subsection{Real time cartographic interaction and its tradeoffs}
% \subsection{An interactive web map as a representation and an interface}
\subsection{Perception and utilization of the interactive map}

% \item Which types of interaction available in the map do people use
% in different types of usage scenarios?
% \item Does dynamic real-time interaction with the map change  % FIXME new term?
% the map user's perception of the mapped phenomenon? If it does, how?
By carrying out the survey I aimed to find out what types of interaction
map users utilize in different types of usage scenarios
and whether dynamic real-time cartographic interaction changes the way
map users perceive the mapped phenomenon.
My results showed that, in the context of map use \textit{as prompted in the questionnaire},
the participants preferred as dynamic as possible interaction with the map.
This was true in two different kinds of tasks,
in addition to the participants rating the more dynamic mode of interaction
more useful than the less dynamic one.
This supports the recognized need for dynamic interfaces in exploring and analysing
spatial data \parencite{eds2008, but2018}.
However, generalizing these findings to answer what mode or style of interaction is
\enquote{the best} is not possible here,
nor is it really a realistic or sensible proposition.
At the level of the map interface, analytical thinking as facilitated by cartographic interaction
consists of immensely diverse \enquote{usage scenarios} and \enquote{tasks}
\parencite{rob2017b, and2010},
all with their respective needs for interaction capabilities with the map
\parencite{rot2013a, rot2015}.
This was also reflected in my results:
In addition to singular map functionalities,
participants highlighted different combinations of map interactions as most useful,
and also mentioned how the value of a given type or style of interaction is
dependent on the usage scenario.  % experienced freedom in interface?

Based on my results, the participants' perception of the mapped phenomenon did change
depending on the mode of interaction they used in examining it.
A key aspect is also that the participants found it harder to decide which one
of two locations is more accessible
when using the more dynamic hovering mode to carry out the assessment.
The increased difficulty of the
comparison is unlikely to be caused by
issues in using the particular mode of interaction:
In general, the participants regarded the map interface as easy to use,
chose to use the hovering mode when given the chance,
and rated the hovering mode as the more useful mode
of interaction with the map.
Instead, the difficulty in forming a decisive opinion with the hovering mode
could be a by-product of the amount of information presented to the user when using it.
All this suggests that a dynamic, real-time, mode of interaction did provide
a different representation of the mapped phenomenon,
fitting directly into the wider discussion within geovisual analytics:
New methods for interfacing with growing spatial datasets are key to the discipline
\parencite{rob2017b, and2010}.
Approaching my results from the perspective of accessibility research,
it should also be noted that making a definitive comparison of accessibility between
two locations is a complex task \parencite{geu2004, lev2020, hu2019}.
The difficulty the participants experienced in making such a comparison
could reflect that a dynamic mode of interaction
better represented this complexity to the map user.
Multiple participants also reported gaining new insight to the phenomenon
having used the interactive map,
which supports the call for dynamic presentation approaches
for better understanding accessibility \parencite{but2018, te2014}.

The main limitations of the survey were the small sample size,
and the fact that the participants were not exactly a representative sample
of the general population.
Young predominantly male adults who had previous knowledge of accessibility
as well as a high level of familiarity with using map interfaces made up most of the sample.
This must be considered in the above interpretation of the results.
Another factor that could affect the results,
especially when it comes to the perceived usefulness of the two modes of interaction,
is the fact that the map interface and data preprocessing have both been
designed primarily for the most dynamic mode of interaction.
In this sense, it could be argued that the comparison between the locked
and hovering modes is not really a fair one:
The data shown in the less dynamic locked mode is still
simplified to the same extent as when using the more dynamic hovering mode,
even if the less dynamic locked mode could technically allow for
more detailed data.


% \subsection{The value of web-based cartographic interaction}
\subsection{Open science, data and presentation}
% inc2012
The \acrshort{ttm} is an extensive showcase of open science --
It is a reproducible method and an openly available, widely applicable, dataset.
However, in many cases, open access to data does not by itself enable
the exploration or understanding of said data \parencite{obr2016}.
As a dataset, the \acrshort{ttm} could be seen as an example of this.
It consists of gigabytes of plain text spread over thousands of files
that need to be further combined with an external statistical grid to enable
geographical presentation of the data.
Merely comprehending the structure of the dataset requires expertise,
proving to be anything but obvious even to university students specializing in the field
(the author included).
Even though the latest iteration of the dataset \parencite{fin2023}
includes an easier-to-map data format with geometries included,
actually mapping the \acrshort{ttm} would still require
desktop \acrshort{gis} software or a programming environment,
coupled with the know-how of using either option.

% communicative tool
In this sense, the study presented here could be seen as a needed addition to the TTM:
a more accessible representation that enables the dataset to be explored by a wide audience
instead of remaining expert-access only.
In general, providing more equal understanding to data is seen as beneficial,
for example promoting data democratization \parencite{awa2020},
fostering innovation \parencite{man2011},
and facilitating equity through engagement and communication \parencite{kra2021}.
The value of the map presentation is further backed up by the survey results,
as they indicate that the map interface worked --
in both usability and conveying information about the mapped phenomenon.

Just as importantly,
the work carried out in this study continues on the track of open and reproducible science,
as laid out by the existing research on the TTM.
All components of this study, as well as the entire map application, are made
with open-source tools and technologies to be as reproducible as possible.
In addition to all source code being openly available,
the components of the map application are
automatically packaged and published as functional container images,
and the deployment of the application is done declaratively.
This means that, instead of relying on a manual effort,
a functional state of the entire system can be automatically and reliably reproduced.
% In addition to the generally acknowledged benefits of open software
% within the wider scientific discussion,
% such an approach has been also called for specifically in the context of
An approach like this can be seen as a direct answer to the call for such software
within the wider scientific discussion:
\enquote{Software to support GeoVisual Analytics should be lightweight,
easily deployable and usable, rather than huge and complex like current GIS}
\parencite[p.~1596]{and2010}.


\subsection{What was not mapped}

A lot. By this I mean that the options in
conceptualizing, implementing and studying an interactive map presentation are innumerable.
In this study, and in the accompanying map application,
the focus was on as-instant-as-possible interaction instead of detailed maps.
The design of the map interface
was based around the client doing only very basic rendering of ready-made data,
with as little as possible additional computation.
An interesting question thus is,
what would the map, and by extension the whole study, look like with different priorities?
In many ways an opposite route to an interactive representation of massive data could be
to supply the client with very detailed data,
and really take advantage of the innovations happening in the client-side visualization space:
For example, an interface could be built around
interaction with the interface style rather than with only the data.

The survey in this study focused mainly on how the map users interacted with the map interface,
and how they perceived the mapped phenomenon through cartographic interaction.
However, little attention was paid to how the map users perceived
the interface and the interaction exchanges themselves.
While this was a conscious decision in this study,
going deeper into this topic --
as opposed to stopping at the often surface level usability metrics such as ease of use --
could reveal new insight about cartographic interaction as a process.
For example, this map provided great freedom in exploring different locations,
but little freedom in exploring different representation styles.
In general, greater freedom in interfaces is often considered to increase user satisfaction and perceived interactivity.
Cartographic interaction can offer many different types of freedoms in interaction,
with the specialty being the ability to manipulate a map and to see the results of interaction through a map.
exposed using many different interaction styles.
Is cartographic interaction through direct map


With different definitions of accessibility?

Sustainability of interaction?

Too much data / too precise data:
Scaling this type of approach to even more data could be difficult.
Can data be too detailed?

% Accessibility? A whole different technology stack and an immense amount of work

What could not yet have been mapped is the future.
Further research:
This highlights the fact that the technical tradeoffs and limitations found in this study
are in no way inherent to cartographic interaction:
Rather, they are technical limitations, that change in time.
