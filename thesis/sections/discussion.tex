\section{Discussion}

Some ideas for discussion:
\begin{itemize}
	\item Pros and cons of real-time interaction:
	Implementing something like the hover mode could be worth it in the sense
	that it provides a genuinely different way to explore massive data.
	Depending on the data being visualized,
	the loss of detail to enable such a presentation could be immense,
	potentially leading to a shallow presentation with little insight to offer.
	\item Too much data / too precise data:
	Scaling this type of approach to even more data could be difficult. Can data be too detailed?
	\item Tradeoffs in the map:
	now priority is on instant interaction instead of detailed maps.
	What would the map, and by extension the whole study, look like with different priorities?
	\item Reproducibility and open science:
	The \acrshort{ttm} is not only an open dataset but also an open method.
	This work continues on that track,
	as every component is completely open-source \& openly available.
	\item Amount of work \& the need for cooperation:
	Mapmaking is a very small fraction of all the work that goes into implementing cartographic interaction,
	yet all the other work must be done, or the map will not exist.
	Cooperation with developers, map users and cartographers is key.
\end{itemize}

% My results showed that...
\subsection{Real time cartographic interaction and its tradeoffs}

Implementing something like the hover mode could be worth it in the sense
that it provides a genuinely different way to explore massive data.
Depending on the data being visualized,
the loss of detail to enable such a presentation could be immense,
potentially leading to a shallow presentation with little insight to offer.

Tradeoffs in the map:
now priority is on instant interaction instead of detailed maps.


\subsection{Technical considerations in producing interactive maps}

Too much data / too precise data:
Scaling this type of approach to even more data could be difficult.
Can data be too detailed?

Amount of work and the need for cooperation:
Mapmaking is a very small fraction of all the work
that goes into implementing cartographic interaction,
yet all the other work must be done,
or the map will not exist.
Cooperation with developers, map users and cartographers is key.

Generalization of results is difficult if not impossible

Difficulty in assessing options: Testing the technology or the implementation.


\subsection{Open source and open science -- measurements, modelling, data and mapping}

% TTM overview: measurements, modelling, dataset
Not only a dataset, the \acrshort{ttm} is an extensive showcase of open science.

% Not complete without representation
In many cases, open access to data does not by itself enable the exploration of said data
\parencite{obr2016}.
It could be argued that the \acrshort{ttm} is a prime example of this.
In its raw, unprocessed, form,
the dataset constitutes gigabytes of data spread over thousands of plain text files.
Merely comprehending the structure of the data requires expertise,
proving to be anything but obvious even to university students specializing in the field
-- the author included.
Furthermore, actually mapping the \acrshort{ttm} would require
desktop \acrshort{gis} software or a programming language,
coupled with the know-how of using either.
An open representation for open data

% The representation from the angle of access to understanding data, reproducibility and open science
This work continues on that track,
as every component is completely open-source \& openly available.


\subsection{What was not mapped}

What would the map, and by extension the whole study, look like with different priorities?

Further research:


\section{Conclusions}
