\section{Data and methods}

This section will focus on the matrix visualisation.

\subsection{Data - Helsinki Region Travel Time Matrix}

General description of the matrix goes here.

\subsection{Methods}

I see the methods necessary to implement the visualisation belonging to two themes:
Methods for figuring out how the map should be
and methods for actually making the map be that way.

For figuring out what the map should be like,
I have (hopefully) at this point already formed some ideas from the background section.
To complement those, and to keep the qualitative aspect of cartography relevant,
interview(s) will be used alongside the development process.

From the technical angle, there are three main components to the map:

\begin{enumerate}
	\item Preprocessing of the matrix to a mappable format
	\item A back-end to serve the matrix
	\item The web map application
\end{enumerate}

See figure \ref{fig:architechture} for my idea of what the architechture might look like.

\begin{figure}[H]
	\centering
	\includegraphics[width=1\textwidth]{images/architechture}
	\caption{Architechture}
	\label{fig:architechture}
\end{figure}

When developing the visualisation, the priority should be on the map application.
However, the development must progress on all components as an iterative process,
where producing a minimal working state should be the first goal.
Something that, to some extent, works, makes discussion about the map possible,
which in turn should keep development progressing towards the right direction.

It should also be noted that the number of techical options for implementing the map is large.
Questions such as which mapping library or UI framework to use,
or how to preprocess the matrix data should be covered here too.
Depending on the need and extent of comparisons between different technologies,
some synthesis could be formed about that too.
