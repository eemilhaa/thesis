\section{Results}

\subsection{Assessment of data preprocessing methods}

% \item Considering rapid real-time interaction,
% what are the performance bottlenecks of
% an interactive web map application,
% and what data preprocessing and simplifying approaches
% are needed to overcome them?
The preprocessing parameters used for
processing the TTM for the finished map application were
a 15-minute isochrone interval and a 60-minute travel time limit.
I reduced the coordinate precision as much as
was possible without impacting the map visually.
This meant a coordinate precision of 5 decimal places,
reduced from the 15-decimal-place GeoJSON default \parencite{geojsonspec}.
File minimizing consisted of removing all whitespace from the GeoJSON files
and applying maximum gzip compression.
Preprocessing the data with these parameters
enabled a responsive enough presentation for exploring all travel modes
with both modes of interaction,
but still kept the map informative, at least from a macro-scale perspective.

Assessing these methods in more detail,
the aggregation of travel time values into 15-minute isochrone polygons was the most
effective method when it came to increasing the responsiveness of the map:
Without the isochrone approach,
the hovering mode would not have been realistic to implement.
However, 15-minute isochrone polygons also
reduced the amount of information on the map greatly,
making detail-oriented analysis of the mapped phenomenon impossible.
Limiting the spatial extent of data based on travel time also had a noticeable effect
in both the responsiveness and information loss,
but its was more dependent on the travel mode.
On slower modes of travel, for example walking,
a travel time maximum of 60 minutes resulted in
much more data being left out than on a travel mode like public transportation or car.
Likewise, the gain in responsiveness was larger on slower travel modes.
When reasoning about the performance bottleneck of the map application,
it should be noted that both these methods
reduced the geometrical complexity of the data greatly,
but also, as a result, affected file-sizes.

Limiting coordinate precision and minimizing the GeoJSON files
reduced the file sizes,
but neither method had a visible effect on map responsiveness.
These methods, with the parameters applied here,
did not result in any information loss on the map,
and did not alter the geometrical complexity of data.

In general, the methods with the most geometrical simplification
resulted in the largest gains in the responsiveness of the map,
while methods focused solely on file sizes did not affect responsiveness.
Given this, in the context of this web map application,
the geometrical complexity of data was the performance bottleneck --
file minimizing alone was not enough for enabling real-time interaction.

See table \ref{tab:preprocessing methods} for
an overview of the assessment of preprocessing methods.

\begin{table}[H]
	\caption{The preprocessing methods used}
	\label{tab:preprocessing methods}
	\centering
	\begin{tabular}{ | L{0.2\textwidth} | L{0.35\textwidth} | L{0.35\textwidth} | }
		\hline
		Method
		& Increase in performance
		& Loss of information
		\\
		\hline
		\hline
		Aggregation into isochrones (15 min interval)
		& Large: The isochronal approach is what makes the real-time interaction possible.
		& Large: 15-minunte isochrones allow for quick overview but make detailed assessment impossible.
		\\
		\hline
		Limiting the maximum travel time (60 min)
		& Noticeable--Large: The largest isochrone polygons are more costly to render.
		Avoiding them improves performance, and on limited hardware / download speed
		is essential to keep the map performant.
		& Noticeable depending on travel mode:
		For walking a limit of 60 minutes means that, on average,
		the isochrones cover 3\% of the total area of the dataset.
		For cycling 30\%, for public transit 27\%, for car 98\%.
		For a given grid cell the average of all modes is 40\%.
		\\
		\hline
		Geometrical simplification
		& Noticeable with limited download speeds:
		Reducing the precision of geometry reduces file sizes approximately by (percentage),  % TODO
		and thus makes data transfer faster.
		No visible effect on rendering performance on tested hardware.
		& None: The decreased precision is in no way visible.
		\\
		\hline
		File optimization (simplifying GeoJSON structure, compression using gzip)
		& Noticeable with limited download speeds:
		This reduces file sizes by (percentage),  % TODO
		and thus makes data transfer faster.
		No visible effect on rendering performance on tested hardware.
		& None: These approaches do not affect the content of the data.
		\\
		\hline
	\end{tabular}
\end{table}



\subsection{Assessment of mapping libraries}

% \item Visual quality of the map
% \item Responsiveness of the map
% \item UI integration capabilities (tested only with React)

Considering its visual quality,
the map contained two main aspects to assess:
The base map and the isochrone polygons drawn on top of it.
When assessing the base map,
Deck.gl and Maplibre were essentially equal as both utilized vector base maps.
This meant seamless zooming and less noticeable base map loading.
Leaflet natively uses raster base maps, which resulted in brief yet noticeable
blurring when the map is zoomed and a more detailed raster is being fetched.
When assessing the rendering of the polygons,
I observed Leaflet and Maplibre producing consistent results,
while Deck.gl produced some inconsistent visual output \parenfig{bug}.
Upon further investigating this together with the Deck.gl maintainers,
the rendering error was reproduced and proven to indeed be an issue with the library \parencite{deckbug} --
More specifically, in its underlying triangulation algorithm \parencite{earcutbug}.

\begin{figure}[H]
	\centering
	\includegraphics[width=\textwidth]{visual/figures/screenshots/bug.png}
	\caption{
		Visual artefacts were occasionally present
		when rendering complex polygons using Deck.gl.
		Here, the same isochrone polygon is rendered using Deck.gl (left)
		and a web based GeoJSON preview tool geojson.io (right).
	}
	\label{fig:bug}
\end{figure}

When assessing the responsiveness of the map interface,
Deck.gl and Maplibre were very similiar.
While both enabled real-time interaction with the map,
I observed Deck.gl to be slightly more responsive when using the hovering mode,
i.e. the most computationally demanding mode of interaction.
Leaflet was not tested to the same extent here,
as the map perfofmance degraded noticeably
even when rendering only the base grid of the dataset.
Both Deck.gl and Maplibre utilized WebGL GPU acceleration,
while Leaflet did not.

All libraries had some way of integrating with React.
Deck.gl provided native integration,
meaning no additional wrapper library was needed,
and React integration did not affect the map function.
Leaflet and Maplibre both had wrapper libraries for
integrating to React applications.
Integrating Leaflet and React using React Leaflet has previously
been observed to produce performance issues \parencite{gaj2023}.
Thus, the performance issues I observed with Leaflet could be an issue of
the wrapper library instead of the mapping library itself.  % TODO move to discussion?
Integrating Maplibre into React with React Map GL
produced a functional map application,
without issues in the development process.

For an overview of the assessment of mapping libraries,
see table \ref{tab:map library comparison}.


\begin{table}[H]
	\caption{Comparison of mapping libraries}
	\label{tab:map library comparison}
	\centering
	\begin{tabular}{ | L{0.1\textwidth} | L{0.25\textwidth} | L{0.25\textwidth} | L{0.25\textwidth} | }
		\hline
		\textbf{Library}
		& \textbf{Quality of visualization}
		& \textbf{Rendering performance}
		& \textbf{Integration with React}
		\\ 
		\hline
		\hline
		Deck.gl
		& Inconsistent rendering of complex polygons, vector tiles supported
		& GPU accelerated (WebGL), most performant of the tested libraries
		& Designed from ground up to work with React
		\\
		\hline
		Leaflet
		& Correct rendering of polygons, Vector tiles possible through plugins
		& No GPU acceleration, least performant of the tested libraries
		& Integration possible with a 3rd party wrapper
		\\
		\hline
		Maplibre
		& Correct rendering of polygons with very rare inconsistencies, vector tiles supported
		& GPU accelerated (WebGL), slightly less performant than deck.gl
		& Integration possible with a 3rd party wrapper
		\\
		\hline
	\end{tabular}
\end{table}


\subsection{Survey on map use}

% \item Which types of interaction available in the map do people use
% in different types of usage scenarios?
% \item Does rapid real-time interaction with the map change
% the map user's perception of the mapped phenomenon? If it does, how?
Refer to appendix \ref{appendix:questionnaire responses} for plots of the responses.

The responses in the first task indicated that participants were
able to use and interpret the map interface accurately.
All but one (1) responses
were on the correct answer options.

When comparing the answers in tasks 2 and 3,
The mode of interaction did affect how the participants perceived the mapped phenomenon:
Most of the participants perceived
the accessibility of the same two locations
differently depending on the mode of interaction used.
When using the hovering mode, the participants reported more difficulty in
performing a definitive comparison between the two locations,
i.e. ranking which location is more accessible.

In tasks 4 and 5,
the participants used the hovering mode the most.
In task 4, when comparing point-like locations,
this difference was not very clear:
12 participants used the clicking mode more while 14 used the hovering mode more.
In task 5, when assessing how accessibility changes between the locations,
a more apparent majority of 23 participants used the hovering mode more,
as opposed to 8 participants using the clicking mode more.
In general,
most of the participants used the hovering mode more,
choosing to use it for both types of tasks.
Still,
task type did affect this,
as the distribution of responses differed based on task type.

When asked about their use of the map interface in general,
Most of the participants rated selection of travel mode as
the most useful functionality available in the map interface.
Hovering mode was rated the second most useful functionality,
and the clicking mode third.
Three (3) participants did not specify any particular functionality,
instead selecting the option \enquote{other}
and detailing their answer as open text.
These participants specified the most useful functionality as:
\begin{itemize}
	\item The combination of all listed functionalities (n = 1)
	\item The combination of the hovering mode and travel mode selection (n = 1)
	\item The combination of clicking and hovering modes, as well as panning and zooming the map view (n = 1)
\end{itemize}
In addition,
all of these elaborations also mentioned
that the usefulness of the functionalities was highly task-dependent.

Most of the participants did not feel that using the map affected
their understanding of the mapped phenomenon.
Those that did feel that their understanding was affected, mentioned:
\begin{itemize}
	\item Gaining new insight on the accessibility of different locations (n = 6)
	\item Gaining new insight on the differences between travel modes (n = 4)
	\item Understanding accessibility differently in general (n = 3)
	\item Understanding the city structure differently (n = 1)
\end{itemize}

A clear majority of the participants found the map interface easy to use.
Those that experienced difficulties, mentioned:
\begin{itemize}
	\item Poor discoverability of the clicking and hovering modes (n = 1)
	\item Issues with using the map with a touchpad (n = 1)
	\item Issues with using the map on a mobile device (n = 1)
	\item Issues with understanding the base map (n = 1)
\end{itemize}

Analysing the responses based on the background information of the participants
did not reveal new patterns in the data.
While I observed subtle differences in the distribution of responses,
no task outcome was altered based on participant background.
It should also be noted that
background-based subsets of the data were often very small (less than 10 responses).
