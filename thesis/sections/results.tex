\section{Results}

\subsection{Assessment of data preprocessing and simplification approaches}

% \item Considering rapid real-time interaction,
% what are the performance bottlenecks of
% an interactive web map application,
% and what data preprocessing and simplifying approaches
% are needed to overcome them?

In the following, I will go over the findings in assessing the preprocessing and simplification approaches.
Of course, all approaches listed here can be applied with varying parameters,
which would change the results.
To provide a summary,
I present the approaches here with the parameters used in processing the data for the finished map application.

The aggregation of travel time values into 15-minute isochrone polygons was the most 
effective preprocessing step when it came to increasing the responsiveness of the map.
However, it also reduced the visual detail of the map greatly.
A travel time maximum of 60 minutes also had a noticeable effect in both these regards,
but its was dependent on the travel mode.
Limiting the extent on slower modes of travel, for example walking,
resulted in much more data being left out than limiting a travel mode such public transportation or car.
The 15-minute isochrone interval and 60-minute travel time limit
were the values that I settled on in the development process --
they produced a responsive enough presentation for exploring all travel modes,
but still kept the map informative, at least from a macro-scale perspective.
When reasoning about the performance bottleneck of the map application,
it should be noted that both these methods reduced the geometrical complexity of the data greatly,
but also, as a result, affected file-sizes.




\begin{table}[H]
	\caption{The preprocessing methods used}
	\label{tab:preprocessing methods}
	\centering
	\begin{tabular}{ | L{0.2\textwidth} | L{0.35\textwidth} | L{0.35\textwidth} | }
		\hline
		Method
		& Increase in performance
		& Loss of information
		\\
		\hline
		\hline
		Aggregation into isochrones (15 min interval)
		& Large: The isochronal approach is what makes the real-time interaction possible.
		& Large: 15-minunte isochrones allow for quick overview but make detailed assessment impossible.
		\\
		\hline
		Limiting the maximum travel time (60 min)
		& Noticeable--Large: The largest isochrone polygons are more costly to render.
		Avoiding them improves performance, and on limited hardware / download speed
		is essential to keep the map performant.
		& Noticeable depending on travel mode:
		For walking a limit of 60 minutes means that, on average,
		the isochrones cover 3\% of the total area of the dataset.
		For cycling 30\%, for public transit 27\%, for car 98\%.
		For a given grid cell the average of all modes is 40\%.
		\\
		\hline
		Geometrical simplification
		& Noticeable with limited download speeds:
		Reducing the precision of geometry reduces file sizes approximately by (percentage),  % TODO
		and thus makes data transfer faster.
		No visible effect on rendering performance on tested hardware.
		& None: The decreased precision is in no way visible.
		\\
		\hline
		File optimization (simplifying GeoJSON structure, compression using gzip)
		& Noticeable with limited download speeds:
		This reduces file sizes by (percentage),  % TODO
		and thus makes data transfer faster.
		No visible effect on rendering performance on tested hardware.
		& None: These approaches do not affect the content of the data.
		\\
		\hline
	\end{tabular}
\end{table}



\subsection{Assessment of mapping libraries}

\begin{table}[H]
	\caption{Comparison of mapping libraries}
	\label{tab:map library comparison}
	\centering
	\begin{tabular}{ | L{0.1\textwidth} | L{0.25\textwidth} | L{0.25\textwidth} | L{0.25\textwidth} | }
		\hline
		\textbf{Library}
		& \textbf{Quality of visualization}
		& \textbf{Rendering performance}
		& \textbf{Integration with React}
		\\ 
		\hline
		\hline
		Deck.gl
		& Inconsistent rendering of complex polygons, vector tiles supported
		& GPU accelerated (WebGL), most performant of the tested libraries
		& Designed from ground up to work with React
		\\
		\hline
		Leaflet
		& Correct rendering of polygons, Vector tiles possible through plugins
		& No GPU acceleration, least performant of the tested libraries
		& Integration possible with a 3rd party wrapper
		\\
		\hline
		Maplibre
		& Correct rendering of polygons with very rare inconsistencies, vector tiles supported
		& GPU accelerated (WebGL), slightly less performant than deck.gl
		& Integration possible with a 3rd party wrapper
		\\
		\hline
	\end{tabular}
\end{table}


\begin{figure}[H]
	\centering
	\includegraphics[width=\textwidth]{visual/figures/screenshots/bug.png}
	\caption{caption \parencite{deckbug}}
	\label{fig:bug}
\end{figure}

\subsection{Survey on map use}

(Response plots in appendices)

Most map users preferred as dynamic as possible interaction with the map,
using the most dynamic mode of interaction when given the choice.
This was the case regardless of the task type (figures \ref{fig:task 4} and \ref{fig:task 5}).

Map users rated selection of travel mode as the most useful way of
interacting with the map (figure \ref{fig:general questions}).
Hovering mode was the second most useful functionality,
and the clicking mode third.

Map interaction did affect how map users perceived the mapped phenomenon.
Most users perceived the accessibility of a location differently depending
on the mode of interaction used (figures \ref{fig:task 2} and \ref{fig:task 3}).

However, most map users did not feel like the map affected
their understanding of accessibility (figure \ref{fig:general questions}).
Those that did, mentioned:

\begin{itemize}
	\item Gaining new insight on the accessibility of different locations (n = 6)
	\item Gaining new insight on the differences between travel modes (n = 4)
	\item Understanding accessibility differently in general (n = 3)
	\item Understanding the city structure differently (n = 1)
\end{itemize}

Overall, the responses indicate that the map worked,
both in the sense of usability and conveying information about the mapped phenomenon.

