\begin{spacing}{1}
\setlength{\parskip}{4pt}

\textbf{Faculty:} Faculty of Science

\textbf{Degree programme:} Master's Programme in Geography

\textbf{Study track:} Geoinformatics

\textbf{Author:} Eemil Haapanen

\textbf{Title:} \mytitle

\textbf{Level:} Master's thesis, 30 credits

\textbf{Month and year:} April 2024  % TODO check

\textbf{Number of pages:} 86 + 11 appendices  % TODO check

\textbf{Keywords:} Cartographic interaction, Interactive map, Map use, User survey, Web mapping

\textbf{Supervisors:} Tuuli Toivonen (PhD), Christoph Fink (PhD), Pyry Kettunen (D.Sc. (Tech.))

\textbf{Where deposited:} University of Helsinki electronic theses library E-thesis/HELDA

\textbf{Additional information:} --

\textbf{Abstract:} Cartographic interaction, the dialogue between a human and a map, is
a process enabling indispensable ways of reasoning with spatial information.
Interactive maps are digital applications,
increasingly often made with web technologies.
Studying and crafting cartographic interaction calls for
user-inclusive studies designed around interactive map use,
also necessitating the assessment of the rapidly evolving technologies enabling interactive maps.

This study combines the technology- and user-centric aspects of cartographic interaction.
I ask what the performance bottlenecks of a web map application are,
and how different web mapping libraries compare as a platform for real-time cartographic interaction.
I also ask how users interact with a highly interactive map interface,
and whether cartographic interaction changes the way they perceive the mapped phenomenon.

Developing a web map application,
a map interface to a massive dataset on spatial accessibility (the Helsinki region Travel Time Matrix),
is central to this study.
I answer my technology-centric questions by assessing the technological aspects of interactive maps
through the development process.
To answer my user-centric questions,
I carry out a user survey (n=31) by combining the web map application with an online questionnaire.

My results show that the geometrical complexity of data was the main factor limiting
map responsiveness, and that notable differences between web mapping libraries existed
in the context or real-time interaction.
Survey participants preferred to use the most dynamic mode of map interaction,
and perceived the mapped phenomenon differently depending on how they interacted with the map.

These results illustrate the dependence between map interface capabilities and technological design choices.
The results also support the wider call for more dynamic map interfaces,
indicating that real-time cartographic interaction can be a functional approach to exploring complex data.
As a whole, the results highlight the need for the ongoing study of both mapping technologies and map use
in order to discover and utilize the potential of cartographic interaction.

\newpage

\textbf{Tiedekunta:} Matemaattis-luonnontieteellinen tiedekunta

\textbf{Koulutusohjelma:} Maantieteen maisteriohjelma

\textbf{Opintosuunta:} Geoinformatiikka

\textbf{Tekijä:} Eemil Haapanen

\textbf{Tutkielman otsikko:} Dynaaminen reaaliaikainen kartografinen interaktio matka-aikamatriisien tutkimisessa --
interaktiivisen web-karttasovelluksen kehitys sekä käyttäjäkysely

\textbf{Tutkielman taso:} Maisterintutkielma, 30 opintopistettä

\textbf{Kuukausi ja vuosi:} Huhtikuu 2024  % TODO check

\textbf{Sivumäärä:} 86 + 11 liitesivua  % TODO check

\textbf{Avainsanat:} Kartografinen interaktiivisuus, interaktiivinen kartta, Kartan käyttäminen, Käyttäjäkysely, Web-kartta

\textbf{Ohjaajat:} Tuuli Toivonen (FT), Christoph Fink (FT), Pyry Kettunen (TkT)

\textbf{Säilytyspaikka:} Helsingin yliopiston avoin julkaisuaineisto E-thesis/HELDA

\textbf{Muita tietoja:} --

\textbf{Tiivistelmä:} Cartographic interaction, the dialogue between a human and a map, is
a process enabling indispensable ways of reasoning with spatial information.
Interactive maps are digital applications,
increasingly often made with web technologies.
Studying and crafting cartographic interaction calls for
user-inclusive studies designed around interactive map use,
also necessitating the assessment of the rapidly evolving technologies enabling interactive maps.

This study combines the technology- and user-centric aspects of cartographic interaction.
I ask what the performance bottlenecks of a web map application are,
and how different web mapping libraries compare as a platform for real-time cartographic interaction.
I also ask how users interact with a highly interactive map interface,
and whether cartographic interaction changes the way they perceive the mapped phenomenon.

Developing a web map application,
a map interface to a massive dataset on spatial accessibility (the Helsinki region Travel Time Matrix),
is central to this study.
I answer my technology-centric questions by assessing the technological aspects of interactive maps
through the development process.
To answer my user-centric questions,
I carry out a user survey (n=31) by combining the web map application with an online questionnaire.

My results show that the geometrical complexity of data was the main factor limiting
map responsiveness, and that notable differences between web mapping libraries existed
in the context or real-time interaction.
Survey participants preferred to use the most dynamic mode of map interaction,
and perceived the mapped phenomenon differently depending on how they interacted with the map.

These results illustrate the dependence between map interface capabilities and technological design choices.
The results also support the wider call for more dynamic map interfaces,
indicating that real-time cartographic interaction can be a functional approach to exploring complex data.
As a whole, the results highlight the need for the ongoing study of both mapping technologies and map use
in order to discover and utilize the potential of cartographic interaction.

\end{spacing}
