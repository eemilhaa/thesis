\begin{spacing}{1}
\setlength{\parskip}{4pt}

\textbf{Faculty:} Faculty of Science

\textbf{Degree programme:} Master's Programme in Geography

\textbf{Study track:} Geoinformatics

\textbf{Author:} Eemil Haapanen

\textbf{Title:} \mytitle

\textbf{Level:} Master's thesis, 30 credits

\textbf{Month and year:} May 2024

\textbf{Number of pages:} 88 + 10 appendices

\textbf{Keywords:}
Cartographic interaction,
Interactive map,
Map use,
User survey,
Web mapping,
Web map application.

\textbf{Supervisors:} Tuuli Toivonen (PhD), Christoph Fink (PhD), Pyry Kettunen (D.Sc. (Tech.))

\textbf{Where deposited:} University of Helsinki electronic theses library E-thesis/HELDA

\textbf{Additional information:} --

\textbf{Abstract:}
Cartographic interaction, the dialogue between a human and a map, is
a process enabling indispensable ways of reasoning with spatial information.
Interactive maps are digital applications,
increasingly often made with web technologies.
Studying and crafting cartographic interaction calls for
user-inclusive studies designed around interactive map use,
also necessitating the assessment of the rapidly evolving technologies enabling interactive maps.

This study combines the technology- and user-centric aspects of cartographic interaction.
I ask what the performance bottlenecks of a web map application are,
and how different web mapping libraries compare as a platform for real-time cartographic interaction.
I also ask how users interact with a highly interactive map interface,
and whether cartographic interaction changes the way they perceive the mapped phenomenon.

Developing a web map application,
a map interface to a massive dataset on spatial accessibility (the Helsinki region Travel Time Matrix),
is central to this study.
I answer my technology-centric questions by assessing the technological aspects of interactive maps
through the development process.
To answer my user-centric questions,
I carry out a user survey (n=31) by combining the web map application with an online questionnaire.

My results show that the geometrical complexity of data,
i.e. the number and detail of geometries to render,
was the main factor limiting map responsiveness.
Notable differences between web mapping libraries existed
in the context of dynamic real-time interaction.
Survey participants preferred to use the most dynamic mode of map interaction,
and perceived the mapped phenomenon differently depending on how they interacted with the map.

These results illustrate the dependence between map interface capabilities and technological design choices
such as data simplification and software selection.
The results also support the wider call for more dynamic map interfaces,
indicating that real-time cartographic interaction can be a functional approach to exploring complex data.
As a whole, the results highlight the need for the ongoing study of both mapping technologies and map use
in order to discover and utilize the potential of cartographic interaction.

\newpage

\textbf{Tiedekunta:} Matemaattis-luonnontieteellinen tiedekunta

\textbf{Koulutusohjelma:} Maantieteen maisteriohjelma

\textbf{Opintosuunta:} Geoinformatiikka

\textbf{Tekijä:} Eemil Haapanen

\textbf{Tutkielman otsikko:} \myfinnishtitle

\textbf{Tutkielman taso:} Maisterintutkielma, 30 opintopistettä

\textbf{Kuukausi ja vuosi:} Toukokuu 2024

\textbf{Sivumäärä:} 88 + 10 liitesivua

\textbf{Avainsanat:}
Interaktiivinen kartta,
Kartan käyttäminen,
Kartografinen vuorovaikutus,
Käyttäjäkysely,
Verkkokarttasovellus.

\textbf{Ohjaajat:} Tuuli Toivonen (FT), Christoph Fink (FT), Pyry Kettunen (TkT)

\textbf{Säilytyspaikka:} Helsingin yliopiston avoin julkaisuaineisto E-thesis/HELDA

\textbf{Muita tietoja:} --

\textbf{Tiivistelmä:}
Kartografinen vuorovaikutus, dialogi kartan ja sen käyttäjän välillä, mahdollistaa
korvaamattomia tapoja tutkia ja analysoida paikkatietoa.
Interaktiiviset kartat ovat digitaalisia sovelluksia,
joita toteutetaan yhä useammin verkkoteknologioilla.
Kartografisen vuorovaikutuksen tutkiminen ja hyödyntäminen vaatii
interaktiivisten karttojen käytön ympärille suunniteltuja käyttäjätutkimuksia,
sekä nopeasti kehittyvien karttateknologioiden arviointia.

Tämä tutkimus yhdistää kartografisen vuorovaikutuksen teknologia- ja käyttäjäkeskeiset näkökulmat.
Kysyn, mitkä tekijät rajoittavat verkkokarttasovelluksen suorituskykyä,
ja miten erilaiset verkkokarttakirjastot toimivat alustana reaaliaikaiselle kartografiselle vuorovaikutukselle.
Kysyn myös, miten käyttäjät käyttävät interaktiivista karttakäyttöliittymää,
ja muuttaako kartografinen vuorovaikutus tapaa, jolla kartan käyttäjä hahmottaa kartoitetun ilmiön.

Keskeinen osa tutkimusta on kehittää interaktiivinen verkkokarttasovellus
massiiviselle paikkatietoaineistolle spatiaalisesta saavutettavuudesta (Pääkaupunkiseudun matka-aikamatriisi).
Vastaan teknologiakeskeisiin kysymyksiini tämän kehitysprosessin kautta.
Käyttäjäkeskeisiin kysymyksiini vastaan suorittamalla käyttäjäkyselyn (n=31),
jossa tutkin verkkokarttasovelluksen käyttöä verkkokyselyllä.

Tulokseni osoittavat, että visualisoitavan aineiston geometrinen monimutkaisuus
oli kartan suorituskyvyn päärajoite, ja että verkkokarttakirjastojen välillä oli
huomattavia eroja reaaliaikaisen vuorovaikutuksen kontekstissa.
Kyselyyn osallistujat suosivat karttakäyttöliittymän dynaamisinta vuorovaikutustapaa,
ja hahmottivat kartoitetun ilmiön eri tavalla riippuen siitä, kuinka he käyttivät karttaa.

Nämä tulokset havainnollistavat riippuvuutta karttakäyttöliittymien ominaisuuksien
ja teknologisten suunnitteluratkaisujen välillä.
Tulokset myös tukevat laajalti tiedostettua tarvetta dynaamisemmille karttakäyttöliittymille,
osoittaen, että reaaliaikainen kartografinen vuorovaikutus voi olla toimiva lähestymistapa suurtenkin
paikkatietoaineistojen tutkimiseen.
Kokonaisuutena tulokset korostavat tarvetta
sekä karttateknologioiden että karttojen käytön kokonaisvaltaiseen tutkimiseen,
jotta interaktiivisten karttojen potentiaalia voitaisiin hyödyntää.

\end{spacing}
