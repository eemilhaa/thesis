\newcommand{\myabstract}{
    \noindent\parindent=0pt\parskip=.5\baselineskip plus 2pt
    \raggedright
    % - Introduction
    % - Research questions
    % - Methods
    %     1) Development of map application
    %     2) Survey
    % - Results
    % - Discussion 

    % In a world overflowing with information,
    % the relevance of maps and cartography is only increasing.
    % Digital interactive map applications

Cartographic interaction, the dialogue between a human and a map, is
a process enabling indispensable ways of reasoning with spatial information.
Interactive maps are digital applications,
increasingly often made with web technologies.
Studying and crafting cartographic interaction calls for
user-inclusive studies designed around interactive map use,
and the assessment of the rapidly evolving technologies enabling interactive maps.

This study combines the technology- and user-centric aspects of cartographic interaction.
I ask what the performance bottlenecks of a web map application are,
and how different web mapping libraries compare as a platform for real-time cartographic interaction.
I also ask how users interact with a highly interactive map interface,
and whether cartographic interaction changes the way they perceive the mapped phenomenon.

Developing a web map application,
a map interface to a massive dataset on spatial accessibility (the Helsinki region Travel Time Matix),
is central to this study.
I answer my technology-centric questions by assessing the technological aspects of interactive maps
through the the development process.
To answer my user-centric questions,
I carry out a survey (n=31) by combining the web map application with an online questionnaire.

My results show that the geometrical complexity of data was the main factor limiting
map responsiveness, and that notable differences between web mapping libraries existed
in the context or real-time interaction.
Survey participants preferred to use the most dynamic mode of map interaction,
and perceived the mapped phenomenon differently depending on how they interacted with the map.

These results illustrate the dependence between map interface capabilities and technological design choices.
The results also support the wider call for more dynamic map interfaces,
indicating that real-time cartographic interaction can be a functional approach to exploring complex data.
As a whole, the results highlight the need for the ongoing study of both mapping technologies and map use
in order to discover and utilize the potential of cartographic interaction.


% From the user-centric perspective,
% 	what are the performance bottlenecks of
% 	a web-based interactive map,
% 	and what data preprocessing and simplifying methods
% 	are needed to overcome them?
% 	\item What differences are there between web mapping libraries
% 	when used as a platform for dynamic real-time cartographic interaction?
% 	\item Which types of user-map interaction do map users utilize
% 	in different types of map usage scenarios?
% 	\item Does dynamic real-time interaction with a map change
% 	the map user's perception of the mapped phenomenon? If it does, how?
% \end{itemize}

}

\begin{spacing}{1}
    \footnotesize
    
    \begin{tabularx}{\linewidth}{|S{X}S{X}|S{X}|S{X}|S{X}|S{X}|}
        \Xhline{3\arrayrulewidth}
        \multicolumn{3}{?>{\hsize=\dimexpr3\hsize-+3\tabcolsep\relax}S{X}|}{
            Tiedekunta -- Fakultet -- Faculty \newline
            Faculty of Science
        } &
        \multicolumn{3}{>{\hsize=\dimexpr3\hsize-+3\tabcolsep\relax}S{X}?}{
            Osasto -- Institution -- Department \newline
            Department of Geosciences and Geography
        } \\
        \hline
        \multicolumn{6}{?>{\hsize=\dimexpr6\hsize-+6\tabcolsep\relax}S{X}?}{
            Tekijä -- Författare -- Author \newline
            Eemil Haapanen
        } \\
        \hline
        \multicolumn{6}{?>{\hsize=\dimexpr6\hsize+10\tabcolsep\relax}S{X}?}{
            Tutkielman otsikko -- Avhandlings titel -- Title of thesis \newline
            TITLE
        } \\
        \hline
        \multicolumn{6}{?>{\hsize=\dimexpr6\hsize+10\tabcolsep\relax}S{X}?}{
            Koulutusohjelma ja opintosuunta -- Utbildningsprogram och studieinriktning -- Programme and study track \newline
            Master's programme in geography, Geoinformatics
        } \\
        \hline
        \multicolumn{2}{?>{\hsize=\dimexpr2\hsize+2\tabcolsep\relax} S{X}|}{
            Tutkielman taso -- Avhandlings nivå -- Level of the thesis \newline
            Master's thesis, 30 credits
        } &
        \multicolumn{2}{>{\hsize=\dimexpr2\hsize+2\tabcolsep\relax} S{X}|} {
            Aika -- Datum -- Date \newline \newline
            MONTH / YEAR
        } &
        \multicolumn{2}{>{\hsize=\dimexpr2\hsize+2\tabcolsep\relax} S{X}?} {
            Sivumäärä -- Sidoantal -- Number of pages \newline
            PAGES + PAGES appendices
        } \\
        \hline    
        \multicolumn{6}{?>{\hsize=\dimexpr6\hsize+10\tabcolsep\relax} S{X}?} {
            Tiivistelmä -- Referat -- Abstract \newline \newline
            \myabstract
        } \\ [400pt]
        \hline
        \multicolumn{6}{?>{\hsize=\dimexpr6\hsize+10\tabcolsep\relax} S{X}?}{
            Avainsanat -- Nyckelord -- Keywords \newline
            KEYWORDS
        } \\
        \hline
        \multicolumn{6}{?>{\hsize=\dimexpr6\hsize-+6\tabcolsep\relax} S{X}?}{
            Säilytyspaikka -- Förvaringställe -- Where deposited \newline
            University of Helsinki electronic theses library E-thesis/HELDA
        } \\
        \hline
        \multicolumn{6}{?>{\hsize=\dimexpr6\hsize-+6\tabcolsep\relax} S{X}?}{
            Muita tietoja -- Övriga uppgifter -- Additional information \newline
        } \\
        \Xhline{3\arrayrulewidth}
    \end{tabularx}
\end{spacing}

