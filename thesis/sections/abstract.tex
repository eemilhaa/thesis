\newcommand{\myabstract}{
    \noindent\parindent=0pt\parskip=.5\baselineskip plus 2pt
    \raggedright
    In a world overflowing with information,
    the importance of maps and cartography is only growing.
    Cartographic presentations make it possible to gain insight from
    information with a spatial component,
    often uncovering patterns otherwise left unnoticed.
    Instead of static, predetermined presentations,
    maps today are increasingly interactive.
    Interactivity enables maps to be used in ways otherwise impossible,
    for example as a tool for exploring and filtering large amounts of data based on
    user input.

    In this thesis I studied how cartographic interaction can be utilized
    in presenting a massive dataset of spatial accessibility,
    The Helsinki region Travel Time Matrix.
    I explored this by implementing a web-based interactive map,
    where the user can explore a simplified version of the dataset in real time.
    With this development process my aim was to find approaches to
    simplify and present the massive dataset in a way that allows for real-time interaction.
    To find out how people use and understand the presentation,
    I carried out a survey (n=31) by
    combining the interactive map with an online questionnaire.

    Aggregation of travel time values into isochronal polygons
    was the most effective approach in simplifying the data,
    enabling real-time interaction with the dataset.
    However, it also drastically reduced the visual detail of the map.
    I found that solutions for web-based rendering of geographic data
    differed in their suitability for real-time interaction in their
    correctness of visualization, user-interface capabilities and responsiveness.
    Map users preferred as dynamic as possible interaction with the map,
    using the most dynamic mode of interaction when given the choice.
    Most users perceived the accessibility of a location
    differently depending on mode of interaction with the map.
    The outputs of the development process are a reproducible, modular and
    open-source system for high-speed rendering of geographical data,
    as well as an overview of the implementation options and tools for such a system.
    The results of the survey indicate that real-time cartographic interaction
    is valuable when exploring massive geographical data.
}

\begin{spacing}{1}
    \footnotesize
    
    \begin{tabularx}{\linewidth}{|S{X}S{X}|S{X}|S{X}|S{X}|S{X}|}
        \Xhline{3\arrayrulewidth}
        \multicolumn{3}{?>{\hsize=\dimexpr3\hsize-+3\tabcolsep\relax}S{X}|}{
            Tiedekunta -- Fakultet -- Faculty \newline
            Faculty of Science
        } &
        \multicolumn{3}{>{\hsize=\dimexpr3\hsize-+3\tabcolsep\relax}S{X}?}{
            Osasto -- Institution -- Department \newline
            Department of Geosciences and Geography
        } \\
        \hline
        \multicolumn{6}{?>{\hsize=\dimexpr6\hsize-+6\tabcolsep\relax}S{X}?}{
            Tekijä -- Författare -- Author \newline
            Eemil Haapanen
        } \\
        \hline
        \multicolumn{6}{?>{\hsize=\dimexpr6\hsize+10\tabcolsep\relax}S{X}?}{
            Tutkielman otsikko -- Avhandlings titel -- Title of thesis \newline
            TITLE
        } \\
        \hline
        \multicolumn{6}{?>{\hsize=\dimexpr6\hsize+10\tabcolsep\relax}S{X}?}{
            Koulutusohjelma ja opintosuunta -- Utbildningsprogram och studieinriktning -- Programme and study track \newline
            Master's programme in geography, Geoinformatics
        } \\
        \hline
        \multicolumn{2}{?>{\hsize=\dimexpr2\hsize+2\tabcolsep\relax} S{X}|}{
            Tutkielman taso -- Avhandlings nivå -- Level of the thesis \newline
            Master's thesis, 30 credits
        } &
        \multicolumn{2}{>{\hsize=\dimexpr2\hsize+2\tabcolsep\relax} S{X}|} {
            Aika -- Datum -- Date \newline \newline
            MONTH / YEAR
        } &
        \multicolumn{2}{>{\hsize=\dimexpr2\hsize+2\tabcolsep\relax} S{X}?} {
            Sivumäärä -- Sidoantal -- Number of pages \newline
            PAGES + PAGES appendices
        } \\
        \hline    
        \multicolumn{6}{?>{\hsize=\dimexpr6\hsize+10\tabcolsep\relax} S{X}?} {
            Tiivistelmä -- Referat -- Abstract \newline \newline
            \myabstract
        } \\ [400pt]
        \hline
        \multicolumn{6}{?>{\hsize=\dimexpr6\hsize+10\tabcolsep\relax} S{X}?}{
            Avainsanat -- Nyckelord -- Keywords \newline
            KEYWORDS
        } \\
        \hline
        \multicolumn{6}{?>{\hsize=\dimexpr6\hsize-+6\tabcolsep\relax} S{X}?}{
            Säilytyspaikka -- Förvaringställe -- Where deposited \newline
            University of Helsinki electronic theses library E-thesis/HELDA
        } \\
        \hline
        \multicolumn{6}{?>{\hsize=\dimexpr6\hsize-+6\tabcolsep\relax} S{X}?}{
            Muita tietoja -- Övriga uppgifter -- Additional information \newline
        } \\
        \Xhline{3\arrayrulewidth}
    \end{tabularx}
\end{spacing}

