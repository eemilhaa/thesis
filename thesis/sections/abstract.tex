\begin{abstract}
\noindent\parindent=0pt\parskip=.5\baselineskip plus 2pt  % 
In a world overflowing with information,
the importance of maps and cartography is only growing.
Cartographic presentations make it possible to gain insight from
information with a spatial component,
often uncovering patterns otherwise left unnoticed.
Instead of static, predetermined presentations,
maps today are increasingly interactive.
Interactivity enables maps to be used in ways otherwise impossible,
for example as a tool for exploring and filtering large amounts of data based on
user input.

In this thesis I studied how cartographic interaction can be utilized
in presenting a massive dataset of spatial accessibility,
The Helsinki region Travel Time Matrix.
I explored this by implementing a web-based interactive map,
where the user can explore a simplified version of the dataset in real time.
With this development process my aim was to find approaches to
simplify and present the massive dataset in a way that allows for real-time interaction.
To find out how people use and understand the presentation,
I carried out a survey by combining the interactive map with an online questionnaire.

Aggregation of travel time values into isochronal polygons
was the most effective approach in simplifying the data,
enabling real-time interaction with the dataset.
However, it also drastically reduced the visual detail of the map.
I found that solutions for web-based rendering of geographic data
differed in their suitability for real-time interaction in their
correctness of visualization, user-interface capabilities and responsiveness.
Map users preferred as dynamic as possible interaction with the map,
using the most dynamic mode of interaction when given the choice.
Most users perceived the accessibility of a location
differently depending on mode of interaction with the map.
The outputs of the development process are a reproducible, modular and
open-source system for high-speed rendering of geographical data,
as well as an overview of the implementation options and tools for such a system.
The results of the survey indicate that real-time cartographic interaction
is valuable when exploring massive geographical data.
\end{abstract}
