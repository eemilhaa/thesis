\section{Background}
A general comprehension of research in both cartography and accessibility is
vital to approaching any of the research questions presented.
In this section I will cover some key themes and advancements in both fields, % advancements?
later relating them more precisely to accessibility visualisations.

% Currently I see the first two research questions as a sort of a background study
% necessary to form a theoretical base
% for the implementation of the interactive matrix visualisation.
% This is why I have placed them in a background section.

% Cartography - How we got to where we are now, what this means for the topic
\subsection{Cartography and maps}

\subsubsection{Definitions}

% maps
% basic overview type stuff, relating information to location
% For most of humanity's time maps...
% What cartography, when
The discipline of cartography is as dynamic as the topics people map,
or the methods they use to map said topics.
Consequently, the term has been difficult to define \parencite{kry1995},
and, even if a definition is agreed upon,
it will always remain a product of the time period
in which it was envisioned \parencite{tyn1992, and1996}.
Currently, as defined by the International Cartography Association (ICA), "Cartography is
the science, art, and technology of making and using maps."
\parencite{ica2019}. Cite \parencite{kra2017} too.  % TODO here or maps? (below)

Much like cartography, the concept of a map is not static.
In a review of various historical writings \textcite{and1996}
found over 321 unique definitions for the word.
Most of these definitions shared the premise that
maps are representations of the surface of the earth,
but, other than that, not many similarities were found.
More current efforts in defining the word range
from lengthy attempts at precise delineation (for example \textcite{ica2003, })
to much shorter definitions (for example \textcite{}).
If the ICA is to be considered an authority, a map is currently
"an abstract visual representation of the geo-environment" \parencite{ica2019}.
Even though quite concise, this definition has gotten its fair share of critique as well
(for example \textcite{lap2021}).

% To understand how we got to these definitions:
To better understand these terms,
and, more importantly, the field of cartography as a whole,
a brief look to the past is needed.
% TODO
% For most of humanitys history maps have been a tool to describe the physical world,
% used for navigation and comprehending places etc
% Nowadays data visualization
% Presenting the world (locations) vs presenting data linked to locations (thematic mapping, \parencite{tyn1992})
% Change, cartography is different now, 1950-1990
\textcite{mac2004} considers two developments especially important
to understanding how the modern cartographic research has shaped up.

% 1950
The first, set in motion by the second world war, was
the shift from thinking of maps as primarily objects of art and graphic design,
to instead viewing them as scientifically dissectible visual representations
with the primary function of conveying information.
Interlinked with the field of psychology,
functional map design drew from the researh on human perception
in an effort to optimize cartographical methods such as symbology.
The goal of these efforts was finding scientifically provable, "objective", rules
for producing as functional as possible maps.

% 1970
% Cartography as graphic communication.
The second was the change in how a map is thought to work.
Instead of an artifact of it's own knowledge,
a map is essentially just a means to trasnport knowledge in a communication system.
So, the knowledge a map reader gains from reading a map is that of the mapper's,
just disseminated through cartographical communication, i.e. the map.
Thinking this way, no new knowledge is constructed by the map reader,  % TODO check this & debunk later
and the acts of mapmaking and reading can be related to constructing and interpreting messages.
Draws heavily from the paradigm of graphical communication,  % useless line?
\textcite{bal1966} propose a term, "graphicacy",
to relate the importance of graphical communication to literacy, articulacy and numeracy.
Among the pioneers of this way of thinking about maps was \textcite{kol1969},
who explicitly regarded maps as communication systems.

However, there have been numerous objections to the paradigm of Cartography as a form of communication.
Firstly, there is the issue of communication versus function:
A mapmaker might not have an intent to communicate any particular message to begin with.

Also, implicit vs explicit:
the message the map reader extracts from a map might differ from the one intended, or


% Moving on from these
Art and science \parencite{mac2004, tyn1992}  % compare these

% 1990
% - More interlinked with GIS
% - critical / qualitative GIS too

% 2010-the present -> interactive
% The change in cartography
% Rise of open source \textcite{pet2015}

\subsubsection{Map design}

\subsubsection{Map types}

\subsubsection{Cartographic interaction}

% The nature of interactive cartography
% More than a map: ui, ux
% What even is an interactive cartographer?
\textcite{rot2013a, rot2013b}

\subsubsection{Web mapping techonolgies}


\subsection{Accessibility}

\subsubsection{Definitions}
Alone, the word simply refers to how easy it is to access something,
with all further meaning derived from
additional explanations and the context in which the word is used.

\subsubsection{Accessibility presentanions / Accessibility in the context of cartography}
% This section focuses on the first research question:

What makes accessibility an unique phenomenon,
especially from a cartographical point of view,
and what requirements this places on visual representations of accessibility.

% This will be a litearture review.

% What are the "normal" (cartography-wise) phenomena, do they exist?

\subsection{Interactive maps and their potential in presenting accessibility}
% Interactivity in general
% The default medium for viewing maps and geographical has long been digital.  % TODO add ref
% However, more and more devices allow also for interaction \parencite{mei2019}
The second research question:

\begin{displayquote}
What value do interactive visualisations hold in the context of accessibility?
(General theory of cartography and especially cartographic interaction,
previous interactive maps and presentations relative to the topic.)
\end{displayquote}

Here I will be doing literatrure review as well,
especially relating to the theory of cartography and interactive maps.
Themes such as critical cartography
and the qualitative nature of maps should be found here too.
When previewing previous accessibility visualisations,
some kind of a systematic comparison of previous work
could be a way to form a synthesis here.
