\section{Background}

% Brief "what is this?" section to explain structure of background
If we are to visualize accessibility,
a general understanding of research in both accessibility and cartography is necessary.
After briefly presenting both fields,
I will relate the two together
providing a more focused synthesis of the aspects most relevant to this thesis.

\begin{itemize}
	\item Define terms more in depth (cartography, map, accessibility...)
	\item Small history recap of the relevant advances in the field of cartography
	\item Cartographic interaction theory
	\item Web mapping advances and technologies
	\item Interactive accessibility presentations and other relevant maps: go over some previous implementations and solutions.
\end{itemize}

% Cartography - How we got to where we are now, what this means for the topic
\subsection{Maps, cartography and cartographic interaction}

\subsubsection{Defining map}
While the meaning of the word map seems obvious,
defining it has been anything but simple for cartographers.
In a review of various historical writings \textcite{and1996}
found 321 unique definitions for the word.
Most of these definitions shared the premise that
maps are representations of the surface of the earth,
but, other than that, not many similarities were found.
More recent efforts in defining the word range
from lengthy attempts at precise delineation
(for example \textcite{ica2003})
to much shorter definitions (for example \textcite{kra2017}).
Currently, the \acrlong{ica} (\acrshort{ica}),
the authoritative international body for cartography,
defines map as \enquote{an abstract visual representation of the geo-environment}
\parencite{ica2019}.

% What map for real?
Even though quite concise,
\acrshort{ica}'s definition has received its fair share of critique as well.
For example, \textcite{lap2021} note that
a map can be a concrete object or have concrete qualities
instead of being strictly abstract,
and be perceived with senses other than vision.
The word \enquote{geo-environment} is problematic too.
It would limit maps to depicting only earth
-- an issue found in many map definitions \parencite{tyn2014} --
as well as introduce possible confusion with themes such as
environmental sustainability \parencite{lap2021}.
As a response, \textcite{lap2021} suggest that,
instead of a given environment or a type of map,
\textit{spatial relationships} should be the starting point of defining what a map is.
Thus, \citeauthor{lap2021} arrive at the following definition:
\enquote{a map is a generalized representation of spatial relationships}.
A few years earlier, \textcite{tyn2014} shared much of the same sentiments,
defining a map as \enquote{a graphic representation that shows spatial relationships}.
While in her definition \citeauthor{tyn2014} specifies maps as something graphic,
both authors are in agreement that maps are representations,
and that representing spatial relationships is what makes a representation a map.

% What does "representation" mean?
\textcite{mac2004}, along the same lines, goes as far as to say

\subsubsection{Cartography as a science}
% What cartography?
% This paragraph could move to the definition section above
It is known that
maps have been made and used in human communities for thousands of years
(for example \textcite{hsu1993, sch2014}).
However, cartography,
not as in the practice of mapmaking but as in an established science,
is much younger.
In fact, \textcite{woo2003, kai2020, cra2018} argue that
the discipline of cartography dates back only to the early 1900s,
since that is when a scientific body of theory on maps started to form.
Before that, it was mainly the mathematical theory on map projections
and the production of topographical maps
that were associated with cartography \parencite{kai2020}.
In the time that cartography has been considered its own science,
change has been a constant in the discourse within the discipline \parencite{mac2004} --
% the discipline has reinvented itself many times.
cartography is as dynamic as the topics that people map,
or the methods they use to map said topics \parencite{tyn1992, tyn2014}.
Consequently,
the term cartography has been difficult to define \parencite{kry1995},
and even if a definition is agreed upon,
it should be considered a product of the time period
in which it was envisioned \parencite{tyn1992, and1996}.
Currently, \acrshort{ica} defines cartography as
\enquote{the science, art, and technology of making and using maps}.
\parencite{ica2019}.

To better understand the discipline of cartography,
it is essential to acknowledge
the field's multidisciplinary, dynamic and multi-paradigm nature.
Often, methods of other disciplines are used in cartography,
and, perhaps even more often,
maps and cartography act as tools for other disciplines \parencite{kai2020}.
What these overlapping disciplines are changes as the field evolves
and the paradigms within cartography shift
\parencite{kai2020, mac2004}.
% TODO list relevant fields -> tell that their relevance changes through paradigm shifts
% Maps and cartography are tools for all spatial sciences
% Mathematics
% Computer science
% Psychology
% Social sciences
At first, as stated earlier,
it was mathematics that was the science most relevant to mapmaking and cartography.
Representing the curved surface of earth  %, a geoid in shape,
on a surface of a different shape, often a plane when maps are concerned,
requires a mathematical transformation \parencite{tyn1992}.
In the context of cartography, this transformation is called a map projection.
The effects the projection has on a map vary by
the properties of the specific transformation used and the scale of the map,
but are always present as distortions introduced to the map \parencite{tyn1992}.
This has direct influence on how a map works, both in the sense of
its visual composition and
how the map reader understands the map \parencite{ker2018}.

% For example producing general purpose maps using a projection intended for navigation purposes.

% 1950
A map only gains meaning when it is read \parencite{gri2017}.
The process of reading a map comprises, for example,
perceiving, judging and reasoning,
often problem-solving and learning too \parencite{mon2002}.
All these are cognitive processes \parencite{apacog},
so it is clear how the field of psychology is fundamentally tied with cartography.
Initially, the link between cartography and psychology
was motivated by a shift in the discourse within cartography:
Like many disciplines post World War II,
cartography strived for credibility through empirical science.
Instead of objects of art and graphic design,
maps should be considered scientifically dissectable representations
with the primary function of conveying information \parencite{rob1952}.
Called functional map design \parencite{mac2004},
this movement in cartography employed scientific methods,
especially those of psychology,
to find objectively provable rules
for producing as functional as possible maps \parencite{mon2002}.
For example,
research to optimize cartographic methods, such as symbology or labelling,
was carried out by utilizing empirical studies on human perception \parencite{mac2004}.

% ken2018, Boa2017, fai2021
In addition to studying and improving the function of maps,
functional map design had an essential role in conceptualizing \textit{how} maps function.
Already present in the work of \textcite{rob1952},
the initial concept of map function was strongly linked to graphical communication.
Drawing inspiration from studies in psychology and information sciences,
this discourse was only strengthened in the following decades
in what can be referred to as the paradigm of cartographic communication \parencite{fai2021}.
The premise of cartographic communication is
that a map is a component in a communication system,
in which its purpose is to transport knowledge to the map reader
\parencite{Boa2017}.
Many varyingly intricate models of cartographic communication exist,
but the main principles are that \parencite{ken2018}:
\begin{enumerate}
	\item The knowledge a map reader gains from reading a map is that of the mapper's,
	just transferred through cartographic communication.
	\item The main purpose of the cartographic method is
	to minimize the loss of information in that transfer.
\end{enumerate}

% With the latter half of the twentieth century,
% the relevance of computer science
While the concept of cartographic communication unified cartographers \parencite{ken2018},
the discipline soon evolved into other directions.
The latter half of the twentieth century saw many developments in computer science
that, in turn, enabled notable advances in cartography:
for example computer-stored spatial data,
computer-assisted spatial modelling, and, eventually,
computer graphics \parencite{kai2020}.
As a result, the function of maps widened
and the paradigm of cartographic communication was soon found quite limiting.
For example, if the acts of making and reading maps were strictly equivalent to
encoding and decoding messages,
no new knowledge would be constructed by the map reader in reading a map \parencite{mac2004}.
As maps started to increasingly be used
as exploratory and analytical devices,
this was not necessarily the most fitting way of thinking of them.
Often the mapmaker and reader were the same person,
and the purpose of the map was not strictly to communicate
but also to facilitate analytical thinking
-- in other words help bring about new knowledge \parencite{tob2000, ant1999, kry1995}.
With its roots in analytical cartograpghy \parencite{tob1976},
this concept of map usage became better known as
cartographic visualization \parencite{ant1999},
a subset of the wider trend of scientific visualization
enabled by computer graphics \parencite{nie1997}.

\enquote{The map is never neutral} \parencite[p.~15]{har1989}.
\citeauthor{har1989}'s view on the nature of maps set in motion
a notable shift in the cartographic discourse
that became known as critical cartography \parencite{cra2018}.
At the core of critical cartography is the notion that
maps can never be objective sources of truth --
rather, they are socially constructed instruments of power
that ultimately reflect the ideologies of their makers \parencite{har1989}.
This sociocultural critique of maps and mapmaking
heavily contrasted the earlier paradigms in cartography.
Previously, the often implicitly authoritative map was assumed neutral,
and the methods by which it was studied were in many cases positivist in nature
\parencite{cra2018, fai2021}.

As one main takeaway it should be recognized that
while I presented the above paragraphs quite linearly,
the multiple paradigms and disciplines relevant to cartography co-exist.
For example,
the role of mathematics is always essential to cartography --
be it the study of map projections \parencite{ker2018},
or the development of the mathematical methods
by which phenomena are represented on a map
\parencite{fra2000}.
Cartographic studies focusing on the empirical study of human perception
have continued and been applied to new types of maps (for example \textcite{col2009}),
and models of map communication still act as a basis for
much of cartographic research \parencite{ken2018}.
The cartographic visualisation paradigm is thriving,
largely enabled by the advances of geographic information systems (\acrshort{gis})
and other platforms that advance the analytical and explorative power of maps
\parencite{kra2017}.
Critical cartography has lead to the acknowledgement of the power of maps,
and a mixed-methods approach to studying and crafting them
\parencite{suc2000, cop2009, pic2008}.


% Multi-paradigm present day

% Social critique




% Draws heavily from the paradigm of graphical communication,  % useless line?
% \textcite{bal1966} propose a term, "graphicacy",
% to relate the importance of graphical communication to literacy, articulacy and numeracy.
% Among the pioneers of this way of thinking about maps was \textcite{kol1969},
% who explicitly regarded maps as communication systems.

% Critique of positivism, parallel to psychophysics
% As the communication model evolved and eventually gained competition,


% Fields relevant in the sense of development of the tools of cartography,
% for example computer science \parencite{mon1985}


% \parencite{kai2020}


% % 1970
% % Cartography as graphic communication.
% The second was the change in how a map is thought to work.

% However, there have been numerous objections to the paradigm of Cartography as a form of communication.
% Firstly, there is the issue of communication versus function:
% A mapmaker might not have an intent to communicate any particular message to begin with.

% Also, implicit vs explicit:
% the message the map reader extracts from a map might differ from the one intended, or


% % Moving on from these
% Art and science \parencite{mac2004, tyn1992}  % compare these

% 1990
% - More interlinked with GIS
% - critical / qualitative GIS too

% 2010-the present -> interactive
% The change in cartography
% Rise of open source \textcite{pet2015}



% Different types of maps

% TODO
% For most of humanitys history maps have been a tool to describe the physical world,
% used for navigation and comprehending places etc
% Nowadays data visualization
% Presenting the world (locations) vs presenting data linked to locations (thematic mapping, \parencite{tyn1992})
% Change, cartography is different now, 1950-1990
\subsubsection{Maps as interfaces}

\parencite{kra2017}
The traditional ‘authoritative’ view of the map being a carefully crafted product
by the cartographer, aimed at visually communicating a complete, mostly static database
of known geographic facts to a user, has turned into a participatory and collaborative perspective.
The map has moved beyond the static window to the world and become an
interactive, mobile, dynamic and collaborative interface between a human, groups of
people and the dynamically evolving environment
% The default medium for viewing maps and geographical has long been digital.  % TODO add ref
% However, more and more devices allow also for interaction \parencite{mei2019}
% \begin{displayquote}

Interactive map as an exploratory and analytical device, new information instead of communication \parencite{gri2017}.


% The nature of interactive cartography
% More than a map: ui, ux
% What even is an interactive cartographer?
% A map is both the for of presentation, as well as the method of interaction
% -> a more complex map means basically a user interface -> need for UI library
\textcite{rot2013a, rot2013b}

\subsection{Crafting interactive map presentations}
\subsubsection{Maps as digital applications}
% What? types of maps & feature scope

\subsubsection{Web maps and mapping technologies}

% Map vs Web gis

% technologies

\subsection{Cartographic presentation of accessibility}

\subsubsection{Accessibility as a phenomenon}
% Difficulty & potential for interaction

Alone, the word accessibility simply refers to how easy it is to access something,
with all further meaning derived from
additional explanations and the context in which the word is used.
Maybe history? Background \parencite{ten2020}

Components: people, transport, activities \parencite{jar2018}

% What makes accessibility a unique phenomenon,


% The previous but from a cartographic point of view.
% What requirements this places on visual representations of accessibility.

\subsubsection{Approaches in visualizing accessibility}

% Interactivity in general
% What value do interactive visualizations hold in the context of accessibility?
% (General theory of cartography and especially cartographic interaction,
% previous interactive maps and presentations relative to the topic.)
% \end{displayquote}

% Here I will be doing literature review as well,
% especially relating to the theory of cartography and interactive maps.
% Themes such as critical cartography
% and the qualitative nature of maps should be found here too.
% When previewing previous accessibility visualizations,
% some kind of systematic comparison of previous work
% could be a way to form a synthesis here.

\begin{figure}[H]
	\centering
	\includegraphics[width=0.6\textwidth]{visual/figures/ttm/isochrone_lines.pdf}
	\caption{Isochrones}
	\label{fig:isochrone lines}
\end{figure}

\begin{figure}[H]
	\centering
	\includegraphics[width=0.6\textwidth]{visual/figures/ttm/isochrone_areas.pdf}
	\caption{Areas between isochrones}
	\label{fig:isochrone areas}
\end{figure}

% TODO
% Mention tradeoffs (detail - speed) -> leads to methods
Crafting any cartographic presentation is much about tradeoffs.
Often these tradeoffs are concerned with the visual composition of the presentation --
what the map can and should try to communicate.

% TODO
Previous presentations can be very precise, but locked to a single place

