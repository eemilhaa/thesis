\section{Background}

% Brief "what is this?" section to explain structure of background
If we are to visualize accessibility,
a general understanding of research in both accessibility and cartography is necessary.
After briefly presenting both fields,
I will relate the two together
providing a more focused synthesis of the aspects most relevant to this thesis.

\begin{itemize}
	\item Define terms more in depth (cartography, map, accessibility...)
	\item Small history recap of the relevant advances in the field of cartography
	\item Cartographic interaction theory
	\item Web mapping advances and technologies
	\item Interactive accessibility presentations and other relevant maps: go over some previous implementations and solutions.
\end{itemize}

% Cartography - How we got to where we are now, what this means for the topic
\subsection{Spatial accessibility}

Alone, the word accessibility simply refers to how easy it is to access something,
with all further meaning derived from
additional explanations and the context in which the word is used.
Maybe history? Background \parencite{ten2020}

Components: people, transport, activities \parencite{jar2018}


\subsection{Cartography and maps}

% maps
% basic overview type stuff, relating information to location
% For most of humanity's time maps...
% What cartography, when
The discipline of cartography is as dynamic as the topics that people map,
or the methods they use to map said topics.
Consequently, the term has been difficult to define \parencite{kry1995},
and, even if a definition is agreed upon,
it will always remain a product of the time period
in which it was envisioned \parencite{tyn1992, and1996}.
Currently, as defined by the \acrlong{ica} (shortened to \acrshort{ica}), "Cartography is
the science, art, and technology of making and using maps."
\parencite{ica2019}.

Much like cartography, the concept of a map is anything but static.
In a review of various historical writings \textcite{and1996}
found over 321 unique definitions for the word.
Most of these definitions shared the premise that
maps are representations of the surface of the earth,
but, other than that, not many similarities were found.
More current efforts in defining the word range
from lengthy attempts at precise delineation (for example \textcite{ica2003})
to much shorter definitions (for example \textcite{kra2017}).
If the \acrshort{ica} is to be considered an authority, a map is currently
\enquote{an abstract visual representation of the geo-environment} \parencite{ica2019}.
Even though quite concise, this definition has received its fair share of critique as well.
For example, \textcite{lap2021} note that
a map can be a concrete object or have concrete qualities
instead of being strictly abstract,
and be perceived with senses other than vision.
The word \enquote{geo-environment} is problematic too,
as it would limit maps to depicting earth
while also introducing possible confusion with themes such as
environmental protection \parencite{lap2021}.
\textcite{lap2021} as well as Years earlier, \textcite{tyn2014} acknowledged similar issues with
the then-current definitions of map.  % TODO then-current?
Her take at defining the word, \enquote{A graphic representation that shows spatial relationships},
is

Even though

% To understand how we got to these definitions:
To better understand these terms,
and, more importantly, the field of cartography as a whole,
a brief look to the past is needed.
% TODO
% For most of humanitys history maps have been a tool to describe the physical world,
% used for navigation and comprehending places etc
% Nowadays data visualization
% Presenting the world (locations) vs presenting data linked to locations (thematic mapping, \parencite{tyn1992})
% Change, cartography is different now, 1950-1990
\textcite{mac2004} considers two developments especially important
to understanding how the modern cartographic research has shaped up.

% 1950
The first, set in motion by the second world war, was
the shift from thinking of maps as primarily objects of art and graphic design,
to instead viewing them as scientifically dissectable visual representations
with the primary function of conveying information.
Interlinked with the field of psychology,
functional map design drew from the research on human perception
in an effort to optimize cartographical methods such as symbology.
The goal of these efforts was finding scientifically provable, "objective", rules
for producing as functional as possible maps.

% 1970
% Cartography as graphic communication.
The second was the change in how a map is thought to work.
Instead of an artefact of its own knowledge,
a map is essentially just a means to transport knowledge in a communication system.
So, the knowledge a map reader gains from reading a map is that of the mapper's,
just disseminated through cartographical communication, i.e. the map.
Thinking this way, no new knowledge is constructed by the map reader,  % TODO check this & debunk later
and the acts of map making and reading can be related to constructing and interpreting messages.
Draws heavily from the paradigm of graphical communication,  % useless line?
\textcite{bal1966} propose a term, "graphicacy",
to relate the importance of graphical communication to literacy, articulacy and numeracy.
Among the pioneers of this way of thinking about maps was \textcite{kol1969},
who explicitly regarded maps as communication systems.

However, there have been numerous objections to the paradigm of Cartography as a form of communication.
Firstly, there is the issue of communication versus function:
A mapmaker might not have an intent to communicate any particular message to begin with.

Also, implicit vs explicit:
the message the map reader extracts from a map might differ from the one intended, or


% Moving on from these
Art and science \parencite{mac2004, tyn1992}  % compare these

% 1990
% - More interlinked with GIS
% - critical / qualitative GIS too

% 2010-the present -> interactive
% The change in cartography
% Rise of open source \textcite{pet2015}

\parencite{kra2017}
The traditional ‘authoritative’ view of the map being a carefully crafted product
by the cartographer, aimed at visually communicating a complete, mostly static database
of known geographic facts to a user, has turned into a participatory and collaborative perspective.
The map has moved beyond the static window to the world and become an
interactive, mobile, dynamic and collaborative interface between a human, groups of
people and the dynamically evolving environment

\subsection{Map interaction and web mapping technologies}

% The nature of interactive cartography
% More than a map: ui, ux
% What even is an interactive cartographer?
\textcite{rot2013a, rot2013b}

\subsection{Cartographic presentation of accessibility}
% % This section focuses on the first research question:

% What makes accessibility a unique phenomenon,
% especially from a cartographical point of view,
% and what requirements this places on visual representations of accessibility.

% What are the "normal" (cartography-wise) phenomena, do they exist?
% Interactivity in general
% The default medium for viewing maps and geographical has long been digital.  % TODO add ref
% However, more and more devices allow also for interaction \parencite{mei2019}
% \begin{displayquote}
% What value do interactive visualizations hold in the context of accessibility?
% (General theory of cartography and especially cartographic interaction,
% previous interactive maps and presentations relative to the topic.)
% \end{displayquote}

% Here I will be doing literature review as well,
% especially relating to the theory of cartography and interactive maps.
% Themes such as critical cartography
% and the qualitative nature of maps should be found here too.
% When previewing previous accessibility visualizations,
% some kind of systematic comparison of previous work
% could be a way to form a synthesis here.

% TODO
% Mention tradeoffs (detail - speed) -> leads to methods
Crafting any cartographical presentation is much about tradeoffs.
Often these tradeoffs are concerned with the visual composition of the presentation --
what the map can and should try to communicate.

% TODO
Previous presentations can be very precise, but locked to a single place

