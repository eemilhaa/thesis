\documentclass{article}
\usepackage{settings}

\title{Communicating accessibility - Interactive visualization of travel-time matrices (thesis  plan)}
\author{Eemil Haapanen}

\begin{document}
\maketitle

\section{Introduction}
% Importance of topic

% accessibility as a phenomenon in general, hard to measure and present
The general starting point for most definitions of accessibility is \enquote{potential of opportunities for interaction} \parencite{han1959}.
To measure this potential, a lot of decisions and simplifications have to be made.

Accessibility is inherently tied to location
Presenting accessibility is hard:
Visualizing all locations is impossible with static maps
Interactive visualizations can get us closer

% how interactivity benefits accessibility visualization (because of how accesibility is)
Because of the 
In a study focusing on the usablity of different accessibility instruments and visualizations,  % accessibility instrument
\textcite{te2014} highlight the importance of real-time interactivity between the map and the map user.
\textcite{but2018} state that for maps to efficiently communicate accessibility,
they should be as flexible and dynamic as possible.
However, \citeauthor{but2018} also note that these qualities are often missing in accessibility visualizations.
Along the same lines, \textcite{paj2021} find modern accessibility visualizations often complicated,
and lacking interactivity and flexibility.

% Communicating accessibility instead of analyzing it

% Interactivity in general
The default medium for viewing maps and geographical has long been digital.  % TODO add ref
However, more and more devices allow also for interaction \parencite{mei2019}

% research questions


\section{Data and methods}

\subsection{Data}
\subsection{Methods}

\printbibliography

\end{document}