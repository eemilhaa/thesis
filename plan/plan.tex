\documentclass{article}
\usepackage{settings}

\title{Communicating accessibility - Interactive visualization of travel-time matrices (thesis  plan)}
\author{Eemil Haapanen}

\begin{document}
\maketitle

\section{Introduction}
% Importance of topic

% accessibility as a phenomenon in general, hard to measure
The general starting point for many definitions of accessibility is
\enquote{potential of opportunities for interaction} \parencite{han1959}.
Depending on the factors thought to be impactful to that potential,
accessibility can be defined and measured in many different ways \parencite{pap2016}.
Even if just considering travel distance or time as a measure of access,
when including all the different things accessibility can be measured in relation to,
the amount of combinations gets out of hand fast \parencite{lev2020}.
In addition, accessibility is inherently tied not only to location
but also time \parencite{jar2018},
meaning every place in every time has a level of access
in relation to every other place \parencite{lev2020}.
All this makes measuring accessibility in a wholistic way a complicated task.

% hard to visualize
Similarily to defining or measuring it, visualizing accessibility is complex.
Accessibility visualizations are often constrained to
displaying access in relation to a limited number of pre-selected places (for example \textcite{wei2018}),
or composing an accessibility index
that can be calculated and mapped for all locations,
in relation to potentially many different things and locations (for example \textcite{kim2019}).
However, more complex accessibility measures tend to lead to
less usable visualizations \parencite{te2014},
while mapping more simple measures could lead to an influx of variations to present.
For example separating different travel modes, times of day or target locations
would multiply the amount of visualizations needed to present an analysis of accessibility.



% how interactivity benefits accessibility visualization (because of how accesibility is)
Visualizing all locations is impossible with static maps
Interactive visualizations can get us closer
Because of the 
In a study focusing on the usablity of different accessibility instruments and visualizations,  % accessibility instrument
\textcite{te2014} highlight the importance of real-time interactivity between the map and the map user.
\textcite{but2018} state that for maps to efficiently communicate accessibility,
they should be as flexible and dynamic as possible. However, \citeauthor{but2018} also note that these qualities are often missing in accessibility visualizations.
Along the same lines, \textcite{paj2021} find modern accessibility visualizations often complicated,
lacking in interactivity and flexibility.


% Interactivity in general ???
The default medium for viewing maps and geographical has long been digital.  % TODO add ref
However, more and more devices allow also for interaction \parencite{mei2019}

% research questions

% Communicating accessibility instead of analyzing it
% How to utilize interactive mapping in communicating accessibility

\section{Data and methods}

\subsection{Data}

\subsection{Methods}

\printbibliography

\end{document}