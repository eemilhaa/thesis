\documentclass[12pt]{article}
\usepackage{settings}

\title{Mapping accessibility - Interactive visualization of travel-time matrices (thesis plan)}
\author{Eemil Haapanen}

\begin{document}
\maketitle

\section{Introduction}
% accessibility as a phenomenon in general, hard to measure
The general starting point for many definitions of accessibility is
\enquote{potential of opportunities for interaction} \parencite{han1959}.
Depending on the factors thought to impact that potential,
accessibility can be defined and measured in many different ways \parencite{pap2016}.
Even if just considering travel distance or time as a measure of access,
when including all the different things accessibility can be measured in relation to,
the amount of combinations gets out of hand fast \parencite{lev2020}.
In addition, accessibility is inherently tied not only to location
but also time \parencite{jar2018},
meaning every place in every time has a level of access
in relation to every other place \parencite{lev2020}.
All this makes measuring accessibility in a wholistic way a complicated task.

% hard to visualize too
Similarily to defining or measuring it, visualizing accessibility is complex.
Accessibility visualizations are often constrained to displaying access in relation to
a limited number of pre-selected places (for example \textcite{wei2018}),
or composing an accessibility index that can be calculated and mapped for all locations,
in relation to potentially many different things and locations (for example \textcite{kim2019}).
However, more complex accessibility measures tend to lead to
less usable presentations \parencite{te2014},
while mapping more simple measures could lead to an influx of variations to present.
For example separating different travel modes, times of day or target locations
would multiply the amount of visualizations needed to present accessibility.

% how interactivity benefits accessibility visualization (because of how accesibility is)
Even if visualizing every aspect of accessibility from everywhere to everywhere is impossible,
interactive visualizations could offer some benefits for presenting accessibility.
After all, a key principle of interactivity in map presentations is
the map user's ability to change the content of the map \parencite{rot2013}.
For visualizing accessibility this could mean, for example,
interactive selection of location or travel mode instead of
a static accessibility index that, to a varying extent, tries to account for everything.
Scientific literature on the topic seems to support the idea of
interactivity in accessibility visualization, even indicating a need for such presentations.
In a study focusing on the usablity of different accessibility instruments and visualizations,  % accessibility instrument
\textcite{te2014} highlight the importance of real-time interactivity between the map and the map user.
\textcite{but2018} state that for maps to efficiently communicate accessibility,
they should be as flexible and dynamic as possible.
However, \textcite{but2018} also note that these qualities are often missing in accessibility visualizations.
Along the same lines, \textcite{paj2021} find modern accessibility visualizations often complicated,
and lacking in interactivity and flexibility.

% research questions
% Communicating accessibility instead of analyzing it
% How to utilize interactive mapping in communicating accessibility
This thesis approaches accessibility from a cartographical angle,
especially in the context of interactive mapping.
There are three main research questions.
Firstly, how should accessibility be presented cartographically?
To answer this, I will go into detail of what makes accessibility an unique phenomenon,
especially from a cartographical point of view,
and what requirements this places on visual representations of accessibility.
Secondly, what value do interactive visualizations hold in the context of accessibility?
Here I will focus on the general theory of cartography and especially cartographic interaction,
also previewing previous interactive maps and presentations relative to the topic.
Thirdly, how should an interactive accessibility presentation be implemented?
The goal here is to develop an interactive web-based presentation of the
Helsinki Region Travel Time Matrix \parencite{ten2020}.
In addition to the insights gained from the previous research questions,
I will utilize an iterative development process with interviews
to keep in touch with the map user's perspective.


% Interactivity in general ???
\section{Background}

\subsection{}
\subsubsection{}
% The default medium for viewing maps and geographical has long been digital.  % TODO add ref
% However, more and more devices allow also for interaction \parencite{mei2019}
\subsection{}

\section{Data and methods}

\subsection{Data}

\subsection{Methods}

\printbibliography

\end{document}