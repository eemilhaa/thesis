\documentclass{article}
\usepackage{settings}

\title{Presenting accessibility on the web - Thesis plan}
\author{Eemil Haapanen}

\begin{document}
\maketitle

\section{Introduction}
% Importance of topic

% accessibility as a phenomenon,

% how interactivity benefits (because of how accesibility is)
\textcite{but2018} state that for maps to efficiently communicate accessibility,
they should be as flexible and dynamic as possible.
\citeauthor{but2018} also note that these qualities are often lacking in accessibility visualizations.
Along the same lines, \textcite{paj2021} find modern accessibility visualizations often too complicated,
and not particularly interactive or flexible.
\textcite{te2014} come to a similar conclusion through interactive workshops,
especially highlighting that usability.  % TODO

% Communicating accessibility instead of analyzing it

% Interactivity in general
The default medium for viewing maps and geographical has long been digital.  % TODO add ref
However, more and more devices allow also for interaction \parencite{mei2019}

% state of the art in the current scientific discussion

\section{Data and methods}

\subsection{Data}
\subsection{Methods}

\printbibliography

\end{document}